\documentclass[11pt]{amsbook}

\makeatletter
\def\@thm#1#2#3{%
  \ifhmode\unskip\unskip\par\fi
  \normalfont
  \trivlist
  \let\thmheadnl\relax
  \let\thm@swap\@gobble
  \let\thm@indent\indent % indent
  \thm@headfont{\bfseries}% heading font boldface // changed
  \thm@notefont{\fontseries\mddefault\upshape}%
  \thm@headpunct{.}% add period after heading
  \thm@headsep 5\p@ plus\p@ minus\p@\relax
  \thm@space@setup
  #1% style overrides
  \@topsep \thm@preskip               % used by thm head
  \@topsepadd \thm@postskip           % used by \@endparenv
  \def\@tempa{#2}\ifx\@empty\@tempa
    \def\@tempa{\@oparg{\@begintheorem{#3}{}}[]}%
  \else
    \refstepcounter{#2}%
    \def\@tempa{\@oparg{\@begintheorem{#3}{\csname the#2\endcsname}}[]}%
  \fi
  \@tempa
}
\makeatother

\renewcommand{\chaptername}{Lecture}

\usepackage[T1]{fontenc}
\usepackage[latin1]{inputenc}
\usepackage{times}
\usepackage{microtype}
\usepackage{amssymb}
\usepackage{a4wide}
\usepackage{graphicx}
%\usepackage{paralist}
%\usepackage{bbm}
\usepackage[pagebackref]{hyperref} 
\hypersetup{pdftitle={\title}, pdfauthor={\author}}
\usepackage{verbatim}
\usepackage{url}
\usepackage[pagebackref]{hyperref}
\hypersetup{pdftitle={\title}, pdfauthor={\author}}

\newcommand{\ba}{{\boldsymbol{a}}}
\newcommand{\balpha}{{\boldsymbol{\alpha}}}
\newcommand{\bb}{{\boldsymbol{b}}}
\newcommand{\bc}{{\boldsymbol{c}}}
\newcommand{\be}{{\boldsymbol{e}}}
\newcommand{\bff}{{\boldsymbol{f}}}
\newcommand{\bnu}{{\boldsymbol{\nu}}}
\newcommand{\bm}{{\boldsymbol{m}}}
\newcommand{\bo}{{\boldsymbol{0}}}
\newcommand{\bp}{{\boldsymbol{p}}}
\newcommand{\bq}{{\boldsymbol{q}}}
\newcommand{\br}{{\boldsymbol{r}}}
\newcommand{\bsigma}{{\boldsymbol{\sigma}}}
\newcommand{\bt}{{\boldsymbol{t}}}
\newcommand{\bv}{{\boldsymbol{v}}}
\newcommand{\bw}{{\boldsymbol{w}}}
\newcommand{\bx}{{\boldsymbol{x}}}
\newcommand{\by}{{\boldsymbol{y}}}
\newcommand{\bz}{{\boldsymbol{z}}}
\newcommand{\bone}{{\boldsymbol{1}}}

\newcommand{\RR}{\mathbb{R}}
\newcommand{\Rgeo}{{\mathbb{R}_{\ge0}}}
\newcommand{\Zgeo}{{\mathbb{Z}_{\ge0}}}
\newcommand{\NN}{\mathbb{N}}
\newcommand{\ZZ}{\mathbb{Z}}
\newcommand{\QQ}{\mathbb{Q}}
\newcommand{\CC}{\mathbb{C}}

\newcommand{\cA}{{\mathcal{A}}}
\newcommand{\cC}{{\mathcal{C}}}
\newcommand{\cD}{{\mathcal{D}}}
\newcommand{\cH}{{\mathcal{H}}}
\newcommand{\cL}{{\mathcal{L}}}
\newcommand{\cO}{{\mathcal{O}}}
\newcommand{\cP}{{\mathcal{P}}}

\newcommand{\scp}[2]{\langle #1,#2\rangle}
\newcommand{\fl}[1]{\left\lfloor #1\right\rfloor}
\newcommand{\ce}[1]{\left\lceil #1\right\rceil}
\newcommand{\rcone}[1]{{{}_\Rgeo\!\left\langle #1\right\rangle}}
\newcommand{\zcone}[1]{{{}_\Zgeo\!\left\langle #1\right\rangle}}

\DeclareMathOperator{\interior}{int}
\DeclareMathOperator{\relint}{relint}
\DeclareMathOperator{\conv}{conv}
\DeclareMathOperator{\cone}{cone}
\DeclareMathOperator{\aff}{aff}
\DeclareMathOperator{\vol}{vol}
\DeclareMathOperator{\dist}{dist}
\DeclareMathOperator{\vertices}{vert}
\DeclareMathOperator{\rang}{rang}
\DeclareMathOperator{\sign}{sign}
\DeclareMathOperator{\pyr}{pyr}
\DeclareMathOperator{\bipyr}{bipyr}
\DeclareMathOperator{\Sl}{Sl}

\DeclareMathOperator{\sgn}{sgn} %signum
\DeclareMathOperator{\ggT}{ggT}

\newcommand{\ojo}[1]{\textsf{\bfseries\boldmath #1}}
\newcommand{\scribe}[1]{\begin{center}\emph{Scribe: #1}\end{center}\bigskip}

\graphicspath{{graphics/}}

\numberwithin{equation}{section}

\newtheorem{theorem}{\textbf{Theorem}}[section]
\newtheorem{lemma}[theorem]{Lemma}
\newtheorem{proposition}[theorem]{Proposition}
\newtheorem{corollary}[theorem]{Corollary}
\newtheorem{conj}[theorem]{Conjecture}
\newtheorem{obs}[theorem]{Observation}
\newtheorem{exercise}[theorem]{Exercise}
\newtheorem{example}[theorem]{Example}
\newtheorem{remark}[theorem]{Remark}

\newtheorem{defn}{\textbf{Definition}}[section]

%\includeonly{lecture3}

\begin{document}

\thispagestyle{empty}

\newcommand{\thisyear}{2013}

\
\vfill
\begin{center}
        \Huge \sffamily\bfseries 
        Discrete and Algorithmic Geometry \thisyear
        \medskip
        (Part 2)

\vspace{2cm}
\LARGE
Julian Pfeifle

\vspace{3cm}

\normalfont\LARGE\sffamily
Version of \today

\vspace{5cm}\
\end{center}

This is the preliminary version of the lecture notes for the second
part of \emph{Discrete and Algorithmic Geometry} (Universitat
Polit�cnica de Catalunya), held in the fall semester of \thisyear\ by Ferran
Hurtado and Julian Pfeifle.

\medskip
These notes are fruit of the collaborative effort of all participating
students, who have taken turns in assembling this text. The name of
each scribe figures in each corresponding section.

\medskip
%The main literature for this course consists of
%\cite{Conway-Sloane-3rd}, \cite{Conway-Strauss08}
%and~\cite{Senechal95}. 

\medskip Suggestions for improvements will always be gladly received
by \texttt{julian.pfeifle@upc.edu}.

\vfill\


% Local Variables: 
% mode: latex
% TeX-master: "dag-upc"
% End: 

\tableofcontents

\chapter{Title of the lecture}

\scribe{Your name here}


% Local Variables: 
% mode: latex
% TeX-master: "dag-upc"
% End: 

\chapter{Asymptotic f-vectors of families of polytopes}

\scribe{Cecilia Gir\'on}

In this section we are going to study the \textit{unimodality conjecture} which says that there exists an $l = P(L)\in\mathbb{N}$ such that $f_0\leq f_1 \leq \cdots\leq f_l \leq f_{l+1}\leq \cdots \leq f_{d-1}\leq f_d$. We woould like to know if it is true. 

First, we define the \textbf{$f$-vector} as the vector of the form $(f_0,f_{1},\cdots,f_{d-1})$ where $f_i$ is as defined before in (\ref{eq1}). We will say that it is a \textbf{flag $f$-vector} $(f_s)_s = [d]$ such that $f_s$ count the number of flags $F_{i_1} \subset F_{i_2} \subset \cdots \subset F_{i_k}$ where $s = \{i_1,i_2,\cdots, i_k \}$ and $\dim F_{i_k} = i_k$ \footnote{You can also read about $cd$-index}.

\bigskip
The unimodal conjecture described before is known to be false for simplical polytopes of dimension $d\geq 19$ and for non-simplicial polytopes of dimension $d \geq 8$. The following conjecture is not known to be false. 

\textbf{Restricted unimodal conjecture (Anders Bjorner)}: 
\begin{eqnarray*}
 f_0\leq f_1\leq \cdots \leq f_{\lfloor \frac{d-1}{4}}\rfloor\\
 f_{\lfloor \frac{3(d-1)}{4}\rfloor}\geq \cdots \geq f_{d-1}
\end{eqnarray*} 

Intuitively we are sure that there is no way this conjecture could be false, but there is not proof of this. We don't even know if $f_k\geq \frac{1}{10000}\min\{f_0,f_{d-1}\}$ is true.

\bigskip

\textbf{Exercises done during the lecture 11/11/2013. Each one includes one}


\section{Operations on polytopes}
\begin{itemize}
\item \textbf{Cartesian (direct) product} $P\times Q$.
\item \textbf{Direct sum} $P^d \oplus q^e \subset \mathbb{R}^{d+e}$.
\item $P*Q\subset \mathbb{R}^{d+e+1}$. It is like $\oplus$ but the subspaces are skew (i.e. affine and they have no point on common). For example $\square^1 *\square^1 = Pyr(P)$.
\end{itemize}

\bigskip
\noindent\textsc{Example}: Given $f_k(P)$, calculate the $k$-th entry of $Pyr(P)$:
\begin{eqnarray*}
f(P)&=&(f_0,f_1,\cdots, f_{d-1}\\
f_k(Pyr(P)) &=& (f_0 +1, f_1 + f_0, f_2+f_1, \cdots, f_{d-2}+ f_{d-3}, f_{d-1}+ f_{d-2}, 1+ f_{d-1})
\end{eqnarray*}
\begin{flushright}
$\clubsuit$
\end{flushright}

\bigskip
\begin{itemize}
\item  \textbf{Connected sum} $P^d\#Q^d$ where $P$ has as simplicial face $f$ and $Q$ has a simplicial face $G$. 
\end{itemize}

This last operation is used to join the asymptotic function $f(\square^d)$ and its dual $f(\diamondsuit ^d)$. To make it work, since $\square^d$ has no triangulations in its faces, it is enough to cut away one vertex and, this way, get a simplex. Merging both functions using the connected sum gives place to a new function which is a non-unimodal function.


% Local Variables: 
% mode: latex
% TeX-master: "dag-upc"
% End: 

%\chapter{Pick's Theorem; Lattice packings of spheres}

\scribe{Miquel Ra\"{i}ch}

\section{Pick's Theorem}
\begin{theorem}[Pick]
Let $P$ be a lattice polygon in the plane ($P$ is closed, convex, simple and its vertices lie in $\ZZ^{2}$).
The area of $P$ is $$A(P)=\vol_{2}P=I+\frac{1}{2}B-1$$where:\\
$\mbox{    }\mbox{    } I=$ number of interior lattice points of $P$ $=\#\left\{(\mbox{int }P)\cap\ZZ^{2}\right\}$,\\
$\mbox{    }\mbox{    } B=$ number of boundary lattice points of $P$ $=\#\left\{\partial{}P\cap\ZZ^{2}\right\}$
\end{theorem}
\begin{proof}\mbox{[part]}\\
\begin{enumerate}
\item Show that Pick's formula is additive: if $P=P_1\cup{}P_2$, then $$I+\frac{1}{2}B-1=\left(I_1+\frac{1}{2}b_1-1\right)+\left(I_2+\frac{1}{2}B_2-1\right)$$
\begin{align*}A(P)=&A(P_1)+A(P_2)\\
I=I(P)=&I_1+I_2+L-2\\
B=B(P)=&B_1+B_2-2L+2\end{align*}
{\center (Principle of Inclusion-Exlusion $\rightarrow$ M\"obius function)\quad\quad\\[0.3cm]

\begin{minipage}{0.6\textwidth}$\lceil$ This proves:\begin{enumerate}\item Pick$(P_1\cup{}P_2)\Leftarrow$ Pick$(P_1)$, Pick$(P_2)$\\
\item Pick$(P_1)\Leftarrow$ Pick$(P_1\cup{}P_2)$, Pick$(P_2)$ $\rfloor$\end{enumerate}\end{minipage}}\\[0.5cm]

\item Prove it for lattice triangles.\end{enumerate}
\end{proof}

\section{Lattice packings of spheres}

A {\textbf{lattice-packing}} of congruent spheres ($\equiv$ same radius) in $\RR^{d}$ is a packing such that the set $Z=\{\mbox{centers of the spheres}\}$ is a lattice $L$ (free abelian group).

Let $\{v_1, \ldots, v_n\}\in\RR^{d}$ be a generating set for $$M=\left[\begin{array}{ccc}v_1 & \cdots & v_n\end{array}\right]\!\mbox{, then }L=\{M\lambda\colon\lambda\in\ZZ^{n}\}$$
$$M=\overbrace{\left.\left[\begin{array}{cc}1 & \frac{1}{2} \\ 0 & \frac{1}{2}\sqrt{3} \end{array}\!\right]\right\}}^{n}\!d\quad\quad\quad\quad\quad{}M\lambda=\left[\begin{array}{ccc}v_1 & \cdots & v_n\end{array}\right]\!\left[\begin{array}{c}\lambda_1 \\ \vdots \\ \lambda_n \end{array}\right]=\lambda_1{}v_1+\cdots+\lambda_n{}v_n$$\\[0.2cm]

Sphere packing $B_0+L=\{B_0+v\colon{}v\in{}L\}=\{B_0+M\lambda\colon\lambda\in\ZZ^{n}\}$
$$\forall{}p,q\in{}P,\mbox{ }\exists{}D_p, D_q\colon{}D_p\cap{}Dq=\emptyset\quad\quad\quad\quad{}p\in{}P\subset\RR^{d}\mbox{ discrete set of points}$$\\[0.1cm]
Voroni cell of $p$ w.r.t. $P$ is$$\mbox{Vor}(P)=\{y\in\RR^{d}\colon\lVert{}y-p\rVert\leq\lVert{}y-q\rVert\mbox{ }\forall{}q\in{}P\}$$
$\lceil$ Georges Voronoi (s. XIX) $\rfloor$\\[0.3cm]
Voronoi cells are intersections of half spaces$$\mbox{Vor}(P)=\bigcap_{q\in{}P}H_q\quad\quad\mbox{ where }H_q=\{y\in\RR^{d}\colon\lVert{}y-p\rVert\leq\lVert{}y-q\rVert\}$$

\begin{defn}\mbox{ }\\
$\mbox{    }\mbox{    } $polyhedron $\equiv^{def}$ intersection of half-spaces\\
$\mbox{    }\mbox{    } $polytope $\equiv^{def}$ convex hull of a finite point set $\overset{\text{FTPT}}{\equiv}$ bounded polyhedron\\[0.1cm]
{\small \upshape (FTPT: Fundamental theorem of polytope theory)}
\end{defn}\mbox{ }

\begin{enumerate}\item any convex hull of a finite point set is an intersection of half-spaces [easy by calculating convex hull].\\
\item any \underline{bounded} intersection of half-spaces is the convex hull of a finite set of points, unless the intersection is empty.\end{enumerate}\mbox{ }

Any lattice is isomorphic to some $\ZZ^n$, as abelian groups, by the map $v\in{}L\leftrightarrow\lambda\in\ZZ^n\colon{}v=M\lambda$

\mbox{ }\\[1.1cm]
[I will put images another day :P]

% Local Variables: 
% mode: latex
% TeX-master: "dag-upc"
% End: 
 

\bibliographystyle{amsalpha}
\bibliography{dag}

\end{document}

%%% Local Variables: 
%%% mode: latex
%%% TeX-master: t
%%% End: 
