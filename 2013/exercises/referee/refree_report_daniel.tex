\documentclass[11pt]{amsart}

\usepackage{a4wide}
\usepackage{paralist}
\usepackage{url}
\usepackage{nopageno}
\usepackage[utf8]{inputenc}
\usepackage{bm}

\newcommand{\cA}{\mathcal{A}}
\newcommand{\cS}{\mathcal{S}}

\begin{document}
\begin{center}
\textbf{\sffamily
   Discrete and Algorithmic Geometry }

\medskip
  Daniel Torres Moral
\end{center}

\bigskip

\begin{center}
\large{
  \textbf{\sffamily Some geometrical aspects of control points for toric patches} 
  
  by Gheorghe Craciun, Luis David García-Puente and Frank Sottile}
  
\small{  Referee report}

\bigskip
\today

\end{center}

\bigskip

\section*{Content}

  This paper talks about Bézier curves and patches. The main purpose of the paper is to state a generalization of the Birch's theorem, which guarantees injectivity in such curves or patches. As the authors say in the introduction, they use some sections to introduce concepts and small results. In fact, a big part of this document is destined to define and explain concepts. On the other hand, the tools used in the proofs are basic ones and it does not seem they use a surprising idea to prove their results.

\section*{Format}

  In my opinion, this article is very well written. For a reader non-expert in the matter, the reading is just in this point where the subject flows lightly without becoming even a bit hard. The structure of the document is simple, clear and logical. Furthermore it is explained in the introduction. The quantity of examples is generous and they are were they should. Similarly happens to the number of figures: they are clear, concise, and they are where is needed to be. Definitions are also clear and formal, without reaching redundancy. But all this is great for a non-expert reader, so maybe for an expert one the reading maybe a bit slow.
  
  In addition, it uses a funny QED symbol, what is always a plus.

\end{document}
