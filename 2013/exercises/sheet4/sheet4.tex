\documentclass[11pt]{amsart}

\usepackage{a4wide}
\usepackage{paralist}
\usepackage{url}
\usepackage{nopageno}
\usepackage{graphicx}
\usepackage{bbm}

\newcommand{\cA}{\mathcal{A}}
\newcommand{\cS}{\mathcal{S}}
\newcommand{\N}{\mathbbm{N}}
\newcommand{\Q}{\mathbbm{Q}}
\newcommand{\R}{\mathbbm{R}}
\newcommand{\Z}{\mathbbm{Z}}

\DeclareMathOperator{\conv}{conv}
\DeclareMathOperator{\LL}{\mathsf{L}}
\DeclareMathOperator{\Ehr}{\mathsf{Ehr}}

\begin{document}
\begin{center}
\textbf{\sffamily
   Discrete and Algorithmic Geometry }

\medskip
   Julian Pfeifle,
   UPC, 2013 \mbox{}
\end{center}

\bigskip

\begin{center}
  \textbf{\sffamily Sheet 4}

\bigskip
\textbf{\sffamily UNDER CONSTRUCTION}
% due on Monday, December 9, 2013

\end{center}

\bigskip
%\enlargethispage{2cm}
%\section*{Reading}

%\begin{enumerate}
%\setlength{\itemsep}{2ex}
%\item Read Lectures 3, 4 from Ziegler's \emph{Lectures on Polytopes}.

%\item Read Sections 5.1, 5.2, 5.3 from Matou\v sek's \emph{Lectures on
%    Discrete Geometry}.
%\end{enumerate}

%\bigskip
%\bigskip
%\section*{Writing}

\begin{enumerate}
\setlength{\itemsep}{1ex}
\item 
    \begin{itemize}
    \item The $(d,k)$-\emph{hypersimplex} is the polytope $\Delta(d,k)=\Box_0^d\cap H_k$, where $\Box_0^d$ is the cube $\Box_0^d = \{x\in\R^d: 0\le x_i\le 1\text{ for all }i\in[d]\}$, and
$H_k=\{x\in\R^d:\sum_{i=1}^d x_i = k\}$.
\item $\Delta'(d,k)=\Box_0^d\cap S_k$, where $S_k$ is the slab $\{x\in\R^d : k-1\le\sum_{i=1}^d x_i \le k\}$.
\item Analogously, define $\Sigma(d,k)=\Delta^d\cap H_k$, where $\Delta^d=\conv\{0,e_1,\dots,e_1+\dots+e_d\}$, and $\Sigma'(d,k) = \Delta^d\cap S_k$.
\item A polytope $P$ is \emph{$\ell$-simplicial} if all $\ell$-dimensional faces of~$P$ are simplices.
\item $P$ is \emph{$\ell$-simple} if all $\ell$-dimensional faces of the polar polytope~$P^\Delta$ are simplices.
    \end{itemize}


\bigskip
\begin{enumerate}
\setlength{\itemsep}{1ex}
\item \emph{``All faces of hypersimplices are hypersimplices''.} True or false? 

\item \emph{``All faces of a $\Sigma(d,k)$ are of the form $\Sigma(d',k')$''.} True or false? 

\item Calculate $f_0$ and $f_{d-1}$ for $\Delta(d,k)$, $\Delta'(d,k)$, $\Sigma(d,k)$ and $\Sigma'(d,k)$.

\item Is there any relationship between $\Delta(d,k)$ and $\Delta'(d',k')$?

\item Is there any relationship between $\Sigma(d,k)$ and $\Sigma'(d',k')$?

\item For each triple $(k,\ell,d)\in\N^3$ with $0\le k,\ell\le d$, decide the truth of the following statements, where $P$ is, in turn, $\Delta(d,k)$, $\Delta'(d,k)$, $\Sigma(d,k)$ and $\Sigma'(d,k)$:

  \begin{inparaenum}
  \item $P$ is $\ell$-simplicial;
  \item $P$ is $\ell$-simple; 
  \item $P$ is $\ell$-neighborly.
  \end{inparaenum}

\emph{Hint:} Use \texttt{polymake} for some small cases, and extrapolate using (a), (b).
\end{enumerate}

\item Let $R$ be an integral rectangle whose edges are parallel to the coordinate axes in $\R^2$, and let $T$ be a rectangular triangle two of whose edges are parallel to the coordinate axes. Show that Pick's Theorem holds for $R$ and $T$.

\item
  \begin{enumerate}
  \item For any $a,b,c,d\in\N$, consider the line segment $S=\conv\{(a,b),(c,d)\}$. Prove that the number of integer points on $S$ is $\gcd(a-c, b-d) + 1$.
  \item For any two fixed positive integers $a,b$, let $T$ be the lattice triangle with vertices $(0,0)$, $(a,0)$, $(0,b)$. 

\begin{enumerate}
\item\label{ex2:i} Compute $\LL_T(t)$ and $\Ehr_T(z)$.
\item Use (i) to derive the following formula for the greatest common divisor of $a$ and $b$:
\[
   \gcd(a,b) 
   \ = \
   2\sum_{k=1}^{b-1}\left\lfloor\frac{ka}{b}\right\rfloor 
   + a + b - ab.
\]
\end{enumerate}
  \end{enumerate}

\end{enumerate}

% \bigskip
 % \bigskip
 % \section*{Software}

 % \begin{enumerate}
 % \setlength{\itemsep}{2ex}
 % \item 
 % \end{enumerate}


\bigskip
\section*{Turning in your work}

Put your answers into a .pdf file. To turn it in, use \texttt{gpg} and the public key \texttt{julian.gpg.pub} in the \texttt{github} repository to create an encrypted copy that is only readable by me. Then commit and push this encrypted file to the repository.
\end{document}
