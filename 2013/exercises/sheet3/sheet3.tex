\documentclass[11pt]{amsart}

\usepackage{a4wide}
\usepackage{paralist}
\usepackage{url}
\usepackage{nopageno}
\usepackage{graphicx}
\usepackage{bbm}

\newcommand{\cA}{\mathcal{A}}
\newcommand{\cS}{\mathcal{S}}
\newcommand{\N}{\mathbbm{N}}
\newcommand{\Q}{\mathbbm{Q}}
\newcommand{\R}{\mathbbm{R}}
\newcommand{\Z}{\mathbbm{Z}}

\DeclareMathOperator{\conv}{conv}

\begin{document}
\begin{center}
\textbf{\sffamily
   Discrete and Algorithmic Geometry }

\medskip
   Julian Pfeifle,
   UPC, 2013 \mbox{}
\end{center}

\bigskip

\begin{center}
  \textbf{\sffamily Sheet 3}

\bigskip
%\textbf{\sffamily UNDER CONSTRUCTION}
 due on Monday, December 2, 2013

\end{center}

\bigskip
%\enlargethispage{2cm}
%\section*{Reading}

%\begin{enumerate}
%\setlength{\itemsep}{2ex}
%\item Read Lectures 3, 4 from Ziegler's \emph{Lectures on Polytopes}.

%\item Read Sections 5.1, 5.2, 5.3 from Matou\v sek's \emph{Lectures on
%    Discrete Geometry}.
%\end{enumerate}

%\bigskip
%\bigskip
\section*{Writing}

\begin{enumerate}
\setlength{\itemsep}{2ex}

\item Compute the polytopality range of the complete bipartite graph $K_{m,n}$ for $m,n\ge3$.

\item Prove or disprove: Any graph on $n$~vertices (connected or not) arises as a subgraph of the $1$-skeleton of a $4$-dimensional polytope.
\end{enumerate}

% \bigskip
 \bigskip
 \section*{Software}

 \begin{enumerate}
 \setlength{\itemsep}{2ex}
 \item Inside your local copy of the \texttt{github} repository of this class, create a \texttt{polymake} extension named \texttt{dag2013-your-name} and copy the original \texttt{polymake} client 

\texttt{apps/polytope/src/join\_polytopes.cc} 

\noindent to the repository:

\texttt{2013/coding/dag2013-your-name/apps/polytope/src/direct\_sum.cc} . 

\noindent Commit only this source file to the repository, and push it to \texttt{github}.

\item Adapt this client to produce the direct sum of two polytopes:
\[
   P\oplus Q
   \ = \
   \conv\big(P\times\{0\}\cup\{0\}\times Q\big),
\]
where $0\in P$, $0\in Q$. Make sure your code is more legible and better formatted than the original, and test your client before pushing the final version! 

\item Use your new client to calculate the $f$-vector of $Q:=\Pi_3\oplus \Pi_3$, where $\Pi_3$ is the $3$-dimensional permutahedron (with $24$~vertices). Write down a single line of \texttt{polymake} code that achieves this calculation. How many facets does $Q$ have? Does your answer agree with the formula for the number of facets we discussed in class?
 \end{enumerate}


\bigskip
\section*{Turning in your work}

Put your answers into a .pdf file. To turn it in, use \texttt{gpg} and the public key \texttt{julian.gpg.pub} in the \texttt{github} repository to create an encrypted copy that is only readable by me. Then commit and push this encrypted file to the repository.
\end{document}
