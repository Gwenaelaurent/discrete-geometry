\documentclass[11pt]{amsart}

\usepackage{a4wide}
\usepackage{paralist}
\usepackage{url}
\usepackage{nopageno}
\usepackage{graphicx}
\usepackage{bbm}

\newcommand{\cA}{\mathcal{A}}
\newcommand{\cS}{\mathcal{S}}
\newcommand{\N}{\mathbbm{N}}
\newcommand{\Q}{\mathbbm{Q}}
\newcommand{\R}{\mathbbm{R}}
\newcommand{\Z}{\mathbbm{Z}}

\DeclareMathOperator{\conv}{conv}
\DeclareMathOperator{\LL}{\mathsf{L}}
\DeclareMathOperator{\Ehr}{\mathsf{Ehr}}

\begin{document}
\begin{center}
\textbf{\sffamily
   Discrete and Algorithmic Geometry }

\medskip
   Julian Pfeifle,
   UPC, 2013 \mbox{}
\end{center}

\bigskip

\begin{center}
  \textbf{\sffamily Sheet 5}

\bigskip
\textbf{\sffamily UNDER CONSTRUCTION}
% due on Monday, December 16, 2013

\end{center}

\bigskip
%\enlargethispage{2cm}
%\section*{Reading}

%\begin{enumerate}
%\setlength{\itemsep}{2ex}
%\item Read Lectures 3, 4 from Ziegler's \emph{Lectures on Polytopes}.

%\item Read Sections 5.1, 5.2, 5.3 from Matou\v sek's \emph{Lectures on
%    Discrete Geometry}.
%\end{enumerate}

%\bigskip
%\bigskip
%\section*{Writing}

\begin{enumerate}
\setlength{\itemsep}{1ex}
\item 

\end{enumerate}

 \bigskip
 % \bigskip
 \section*{Software}

 \begin{enumerate}
   \setlength{\itemsep}{2ex}
 \item To test the code in \texttt{face\_selector.cc} we wrote in class, complete the skeleton file \texttt{selected\_face.cc} to a \texttt{polymake} client that outputs the (indices of the vertices on the) minimal face selected by a linear objective function on a given polytope. 
\begin{enumerate}
\item Test your two programs. For example, using \verb|$p=cube(3);|,
  the command 
\begin{verbatim}
print selected_face($p, face_selector($p, new Set([0,1])));
\end{verbatim}
should return $\{0,1\}$; while 
\begin{verbatim}
print selected_face($p, face_selector($p, new Set([0,7])));
\end{verbatim}
should return  $\{0,1,2,3,4,5,6,7\}$. Can you think of more, meaningful tests?

\item Use your new client to calculate \verb|selected_face($p,$a)|, where \verb|$p=cube($d)|, \verb|$a=face_selector($p,new Set([0,1]))|, and \verb|$d|${}\ge3$ varies. How large can you make \verb|$d| and still get an answer in 10~minutes of computation? What (if anything) changes if instead you use \verb|$p=polarize(center(cyclic($d,2*$d)))|? 
\end{enumerate}
 \end{enumerate}


\bigskip
\section*{Turning in your work}

Put your answers into a .pdf file. To turn it in, use \texttt{gpg} and the public key \texttt{julian.gpg.pub} in the \texttt{github} repository to create an encrypted copy that is only readable by me. Then commit and push this encrypted file to the repository.
\end{document}
