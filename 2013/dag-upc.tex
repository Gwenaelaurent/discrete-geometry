\documentclass[11pt]{amsbook}

\makeatletter
\def\@thm#1#2#3{%
  \ifhmode\unskip\unskip\par\fi
  \normalfont
  \trivlist
  \let\thmheadnl\relax
  \let\thm@swap\@gobble
  \let\thm@indent\indent % indent
  \thm@headfont{\bfseries}% heading font boldface // changed
  \thm@notefont{\fontseries\mddefault\upshape}%
  \thm@headpunct{.}% add period after heading
  \thm@headsep 5\p@ plus\p@ minus\p@\relax
  \thm@space@setup
  #1% style overrides
  \@topsep \thm@preskip               % used by thm head
  \@topsepadd \thm@postskip           % used by \@endparenv
  \def\@tempa{#2}\ifx\@empty\@tempa
    \def\@tempa{\@oparg{\@begintheorem{#3}{}}[]}%
  \else
    \refstepcounter{#2}%
    \def\@tempa{\@oparg{\@begintheorem{#3}{\csname the#2\endcsname}}[]}%
  \fi
  \@tempa
}
\makeatother

\renewcommand{\chaptername}{Lecture}


\usepackage[T1]{fontenc}
\usepackage[latin1]{inputenc}
\usepackage{times}
\usepackage{microtype}
\usepackage{amssymb}
\usepackage{a4wide}
\usepackage{graphicx}
\usepackage{paralist}
\usepackage{bbm}
\usepackage{color}
\usepackage{verbatim}
\usepackage{url}
\usepackage[pagebackref]{hyperref}
\hypersetup{pdftitle={\title}, pdfauthor={\author}}
\usepackage{xypic}

\usepackage[sectionbib]{natbib}
\usepackage[sectionbib]{chapterbib}

\usepackage{amsmath}
%\usepackage{tikz}
%\usetikzlibrary{matrix,arrows,decorations.pathmorphing}

\newcommand{\ba}{{\boldsymbol{a}}}
\newcommand{\balpha}{{\boldsymbol{\alpha}}}
\newcommand{\bb}{{\boldsymbol{b}}}
%\newcommand{\bc}{{\boldsymbol{c}}}
\newcommand{\be}{{\boldsymbol{e}}}
\newcommand{\bff}{{\boldsymbol{f}}}
\newcommand{\bnu}{{\boldsymbol{\nu}}}
\newcommand{\bm}{{\boldsymbol{m}}}
\newcommand{\bo}{{\boldsymbol{0}}}
\newcommand{\bp}{{\boldsymbol{p}}}
\newcommand{\bq}{{\boldsymbol{q}}}
\newcommand{\br}{{\boldsymbol{r}}}
\newcommand{\bsigma}{{\boldsymbol{\sigma}}}
\newcommand{\bt}{{\boldsymbol{t}}}
\newcommand{\bv}{{\boldsymbol{v}}}
\newcommand{\bw}{{\boldsymbol{w}}}
\newcommand{\bx}{{\boldsymbol{x}}}
\newcommand{\by}{{\boldsymbol{y}}}
\newcommand{\bz}{{\boldsymbol{z}}}
\newcommand{\bone}{{\boldsymbol{1}}}

\newcommand{\RR}{\mathbb{R}}
\newcommand{\Rgeo}{{\mathbb{R}_{\ge0}}}
\newcommand{\Zgeo}{{\mathbb{Z}_{\ge0}}}
\newcommand{\NN}{\mathbb{N}}
\newcommand{\ZZ}{\mathbb{Z}}
\newcommand{\QQ}{\mathbb{Q}}
\newcommand{\CC}{\mathbb{C}}

\newcommand{\cA}{{\mathcal{A}}}
\newcommand{\cC}{{\mathcal{C}}}
\renewcommand{\cD}{{\mathcal{D}}}
\renewcommand{\cH}{{\mathcal{H}}}
\renewcommand{\cL}{{\mathcal{L}}}
\newcommand{\cO}{{\mathcal{O}}}
\newcommand{\cP}{{\mathcal{P}}}

\newcommand{\cU}{\mathcal{U}}
\newcommand{\tU}{\tilde{\mathcal{U}}}
\newcommand{\tV}{\tilde{\mathcal{V}}}

\newcommand{\scp}[2]{\langle #1,#2\rangle}
\newcommand{\fl}[1]{\left\lfloor #1\right\rfloor}
\newcommand{\ce}[1]{\left\lceil #1\right\rceil}
\newcommand{\rcone}[1]{{{}_\Rgeo\!\left\langle #1\right\rangle}}
\newcommand{\zcone}[1]{{{}_\Zgeo\!\left\langle #1\right\rangle}}

\newcommand{\bn}{{\color{blue}\texttt{\bfseries n}}}
\newcommand{\bc}{{\color{blue}$\boldsymbol{\circ}$}}
\newcommand{\rn}{{\color{red}\texttt{\bfseries n}}}
\newcommand{\rs}{{\color{red}\texttt{\bfseries *}}}
\newcommand{\rt}{{\color{red}$\boldsymbol{\times}$}}
\newcommand{\bB}{{\color{blue}\texttt{\bfseries B}}}
\newcommand{\rR}{{\color{red}\texttt{\bfseries R}}}

\newcommand{\crb}[1]{{\color{red}\texttt{\bfseries{#1}}}}
\newcommand{\cbb}[1]{{\color{blue}\texttt{\bfseries{#1}}}}

\DeclareMathOperator{\interior}{int}
\DeclareMathOperator{\relint}{relint}
\DeclareMathOperator{\conv}{conv}
\DeclareMathOperator{\cone}{cone}
\DeclareMathOperator{\aff}{aff}
\DeclareMathOperator{\vol}{vol}
\DeclareMathOperator{\dist}{dist}
\DeclareMathOperator{\vertices}{vert}
\DeclareMathOperator{\rang}{rang}
\DeclareMathOperator{\sign}{sign}
\DeclareMathOperator{\pyr}{pyr}
\DeclareMathOperator{\bipyr}{bipyr}
\DeclareMathOperator{\Sl}{Sl}
\DeclareMathOperator{\im}{Im}
\DeclareMathOperator{\colspan}{colspan}
\DeclareMathOperator{\rank}{rank}

%%%%%%%%%%%%%%%%%%%%%%%%%%%%%%%%%%%%%%%%Ferran%%%%%%%%%%%%%%%%%%%%%%%%%%%%%%%%%%%%%%%%%%%%%%%%%%%%%%%%%%%%%%%%%%%%%%%%%%%%%%%
\newcommand{\Proj}{\mathbb{P}}
\DeclareMathOperator{\id}{id}
\DeclareMathOperator{\Vor}{Vor}
\newcommand{\HH}{\mathbb{H}}

\DeclareMathOperator{\sgn}{sgn} %signum
\DeclareMathOperator{\ggT}{ggT}

\newcommand{\ojo}[1]{\textsf{\bfseries\boldmath #1}}
\newcommand{\scribe}[1]{\begin{center}\emph{Scribe: #1}\end{center}\bigskip}

\graphicspath{{graphics/}}

\numberwithin{equation}{chapter}

\newtheorem{theorem}{\textbf{Theorem}}[chapter]
\newtheorem{lemma}[theorem]{Lemma}
\newtheorem{proposition}[theorem]{Proposition}
\newtheorem{corollary}[theorem]{Corollary}
\newtheorem{conj}[theorem]{Conjecture}
\newtheorem{exercise}[theorem]{Exercise}
\newtheorem{example}[theorem]{Example}
\newtheorem{claim}[theorem]{Claim}

\theoremstyle{definition}
\newtheorem{definition}[theorem]{Definition}
\newtheorem{defn}[theorem]{Definition}
\newtheorem{remark}[theorem]{Remark}
\newtheorem{observation}[theorem]{Observation}
\newtheorem{obs}[theorem]{Observation}

% \includeonly{lecture3}

\begin{document}

\thispagestyle{empty}

\newcommand{\thisyear}{2013}

\
\vfill
\begin{center}
        \Huge \sffamily\bfseries 
        Discrete and Algorithmic Geometry \thisyear
        \medskip
        (Part 2)

\vspace{2cm}
\LARGE
Julian Pfeifle

\vspace{3cm}

\normalfont\LARGE\sffamily
Version of \today

\vspace{5cm}\
\end{center}

This is the preliminary version of the lecture notes for the second
part of \emph{Discrete and Algorithmic Geometry} (Universitat
Polit�cnica de Catalunya), held in the fall semester of \thisyear\ by Ferran
Hurtado and Julian Pfeifle.

\medskip
These notes are fruit of the collaborative effort of all participating
students, who have taken turns in assembling this text. The name of
each scribe figures in each corresponding section.

\medskip
%The main literature for this course consists of
%\cite{Conway-Sloane-3rd}, \cite{Conway-Strauss08}
%and~\cite{Senechal95}. 

\medskip Suggestions for improvements will always be gladly received
by \texttt{julian.pfeifle@upc.edu}.

\vfill\


% Local Variables: 
% mode: latex
% TeX-master: "dag-upc"
% End: 

\tableofcontents

\chapter{Title of the lecture}

\scribe{Your name here}


% Local Variables: 
% mode: latex
% TeX-master: "dag-upc"
% End: 

\chapter{Asymptotic f-vectors of families of polytopes}

\scribe{Cecilia Gir\'on}

In this section we are going to study the \textit{unimodality conjecture} which says that there exists an $l = P(L)\in\mathbb{N}$ such that $f_0\leq f_1 \leq \cdots\leq f_l \leq f_{l+1}\leq \cdots \leq f_{d-1}\leq f_d$. We woould like to know if it is true. 

First, we define the \textbf{$f$-vector} as the vector of the form $(f_0,f_{1},\cdots,f_{d-1})$ where $f_i$ is as defined before in (\ref{eq1}). We will say that it is a \textbf{flag $f$-vector} $(f_s)_s = [d]$ such that $f_s$ count the number of flags $F_{i_1} \subset F_{i_2} \subset \cdots \subset F_{i_k}$ where $s = \{i_1,i_2,\cdots, i_k \}$ and $\dim F_{i_k} = i_k$ \footnote{You can also read about $cd$-index}.

\bigskip
The unimodal conjecture described before is known to be false for simplical polytopes of dimension $d\geq 19$ and for non-simplicial polytopes of dimension $d \geq 8$. The following conjecture is not known to be false. 

\textbf{Restricted unimodal conjecture (Anders Bjorner)}: 
\begin{eqnarray*}
 f_0\leq f_1\leq \cdots \leq f_{\lfloor \frac{d-1}{4}}\rfloor\\
 f_{\lfloor \frac{3(d-1)}{4}\rfloor}\geq \cdots \geq f_{d-1}
\end{eqnarray*} 

Intuitively we are sure that there is no way this conjecture could be false, but there is not proof of this. We don't even know if $f_k\geq \frac{1}{10000}\min\{f_0,f_{d-1}\}$ is true.

\bigskip

\textbf{Exercises done during the lecture 11/11/2013. Each one includes one}


\section{Operations on polytopes}
\begin{itemize}
\item \textbf{Cartesian (direct) product} $P\times Q$.
\item \textbf{Direct sum} $P^d \oplus q^e \subset \mathbb{R}^{d+e}$.
\item $P*Q\subset \mathbb{R}^{d+e+1}$. It is like $\oplus$ but the subspaces are skew (i.e. affine and they have no point on common). For example $\square^1 *\square^1 = Pyr(P)$.
\end{itemize}

\bigskip
\noindent\textsc{Example}: Given $f_k(P)$, calculate the $k$-th entry of $Pyr(P)$:
\begin{eqnarray*}
f(P)&=&(f_0,f_1,\cdots, f_{d-1}\\
f_k(Pyr(P)) &=& (f_0 +1, f_1 + f_0, f_2+f_1, \cdots, f_{d-2}+ f_{d-3}, f_{d-1}+ f_{d-2}, 1+ f_{d-1})
\end{eqnarray*}
\begin{flushright}
$\clubsuit$
\end{flushright}

\bigskip
\begin{itemize}
\item  \textbf{Connected sum} $P^d\#Q^d$ where $P$ has as simplicial face $f$ and $Q$ has a simplicial face $G$. 
\end{itemize}

This last operation is used to join the asymptotic function $f(\square^d)$ and its dual $f(\diamondsuit ^d)$. To make it work, since $\square^d$ has no triangulations in its faces, it is enough to cut away one vertex and, this way, get a simplex. Merging both functions using the connected sum gives place to a new function which is a non-unimodal function.


% Local Variables: 
% mode: latex
% TeX-master: "dag-upc"
% End: 

\chapter*{Scribes 2013}

\section{Alex Alvarez}
I have done a double degree in Informatics Engineering and Mathematics at the UPC and during these years I have participated in programming contests both individually and representing the UPC. I am mainly interested in Algorithms and Data Structures and I would like to start a PhD in those topics next year, but I also enjoy learning about Discrete Mathematics, in particular Combinatorics and Graph Theory.


\section{Cecilia Gir\'on Albert}

I've got my Degree in Mathematics at the Universidad Aut\'onoma de Madrid, which I completed my fifth and sixth semesters 
at the University of Jyv\"askyla (Finland) as an Erasmus student. My degree has been mainly focused on subjects like 
analysis, statistics and numerical methods, although I am more interested in algebra and graph theory. 

I decided to study the Master in Advanced Mathematics and Mathematical Engineering to keep developing my mathematics skills 
in some theoretical subjects that can be applied in real life problems. Therefore, I believe that this course is a great 
opportunity to learn more about algorithmic and computing science and even more importantly, it may help me to find out 
what field I would like to focus on in the future.

\section{Anna Somoza}

I have recently finished a degree in Mathematics at the Universitat Polit\`ecnica de Catalunya. During this degree I developed a great interest in Algebra fields. In particular, I took the optional subjects \emph{Algebraic Geometry}, \emph{Algebraic Topology} and \emph{Galois Theory} and I wrote my Final Degree Thesis on a topic of Number Theory.

Now I'm taking the Master in Advanced Mathematics and Mathematical Engineering to develop my knowledge in these and other fields, and I my aim is to start a PhD in Number Theory next year. I enroled this subject because I have allways liked both computer science and geometry, and it seemed to be interesting. Therefore, I would be interested in the topic related to algebraic geometry.

\section{Daniel Torres}

My name is Daniel Torres and I am graduate in mathematics in UPC. Along the degree I have developed much interest in fields of topology, algebra and geometry, and some loathing to study (not to programming) numerical methods and modelling. I decided study this master, and particularly this subject, for expand my knowledge about my interests.

More concretely, I am doing this course with the hope it shows me about geometry.




% Local Variables: 
% mode: latex
% TeX-master: "dag-upc"
% End: 


\bibliographystyle{plainnat}
\bibliography{dag}

\renewcommand{\chaptername}{Chapter}
\chapter*{Papers to referee}


\bibliographystyle{amsplain}
\bibliography{papers-to-referee}

\nocite{donoho}
\nocite{eppstein-ziegler-fat-4-polytopes}
\nocite{friedman-graph-simple-polytope}
\nocite{gebert-sturmfels-theobald}
\nocite{haase-schicho}
\nocite{sanyal}
\nocite{schwartz-ziegler}
\nocite{sottile}
\nocite{szabo-welzl}
\nocite{welzl}

% Local Variables: 
% mode: latex
% TeX-master: "dag-upc"
% End: 


\end{document}

%%% Local Variables: 
%%% mode: latex
%%% TeX-master: t
%%% End: 
