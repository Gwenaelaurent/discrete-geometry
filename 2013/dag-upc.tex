\documentclass[11pt]{amsbook}

\makeatletter
\def\@thm#1#2#3{%
  \ifhmode\unskip\unskip\par\fi
  \normalfont
  \trivlist
  \let\thmheadnl\relax
  \let\thm@swap\@gobble
  \let\thm@indent\indent % indent
  \thm@headfont{\bfseries}% heading font boldface // changed
  \thm@notefont{\fontseries\mddefault\upshape}%
  \thm@headpunct{.}% add period after heading
  \thm@headsep 5\p@ plus\p@ minus\p@\relax
  \thm@space@setup
  #1% style overrides
  \@topsep \thm@preskip               % used by thm head
  \@topsepadd \thm@postskip           % used by \@endparenv
  \def\@tempa{#2}\ifx\@empty\@tempa
    \def\@tempa{\@oparg{\@begintheorem{#3}{}}[]}%
  \else
    \refstepcounter{#2}%
    \def\@tempa{\@oparg{\@begintheorem{#3}{\csname the#2\endcsname}}[]}%
  \fi
  \@tempa
}
\makeatother

\renewcommand{\chaptername}{Lecture}


\usepackage[T1]{fontenc}
\usepackage[latin1]{inputenc}
\usepackage{times}
\usepackage{microtype}
\usepackage{amssymb}
\usepackage{a4wide}
\usepackage{graphicx}
\usepackage{paralist}
\usepackage{bbm}
\usepackage{color}
\usepackage{verbatim}
\usepackage{url}
\usepackage[pagebackref]{hyperref}
\hypersetup{pdftitle={\title}, pdfauthor={\author}}
\usepackage{xypic}

\usepackage[sectionbib,numbers]{natbib}
\usepackage[sectionbib]{chapterbib}

\usepackage{amsmath}
%\usepackage{tikz}
%\usetikzlibrary{matrix,arrows,decorations.pathmorphing}

\newcommand{\ba}{{\boldsymbol{a}}}
\newcommand{\balpha}{{\boldsymbol{\alpha}}}
\newcommand{\bb}{{\boldsymbol{b}}}
%\newcommand{\bc}{{\boldsymbol{c}}}
\newcommand{\be}{{\boldsymbol{e}}}
\newcommand{\bff}{{\boldsymbol{f}}}
\newcommand{\bnu}{{\boldsymbol{\nu}}}
\newcommand{\bm}{{\boldsymbol{m}}}
\newcommand{\bo}{{\boldsymbol{0}}}
\newcommand{\bp}{{\boldsymbol{p}}}
\newcommand{\bq}{{\boldsymbol{q}}}
\newcommand{\br}{{\boldsymbol{r}}}
\newcommand{\bsigma}{{\boldsymbol{\sigma}}}
\newcommand{\bt}{{\boldsymbol{t}}}
\newcommand{\bv}{{\boldsymbol{v}}}
\newcommand{\bw}{{\boldsymbol{w}}}
\newcommand{\bx}{{\boldsymbol{x}}}
\newcommand{\by}{{\boldsymbol{y}}}
\newcommand{\bz}{{\boldsymbol{z}}}
\newcommand{\bone}{{\boldsymbol{1}}}

\newcommand{\RR}{\mathbb{R}}
\newcommand{\Rgeo}{{\mathbb{R}_{\ge0}}}
\newcommand{\Zgeo}{{\mathbb{Z}_{\ge0}}}
\newcommand{\NN}{\mathbb{N}}
\newcommand{\ZZ}{\mathbb{Z}}
\newcommand{\QQ}{\mathbb{Q}}
\newcommand{\CC}{\mathbb{C}}

\newcommand{\cA}{{\mathcal{A}}}
\newcommand{\cC}{{\mathcal{C}}}
\renewcommand{\cD}{{\mathcal{D}}}
\renewcommand{\cH}{{\mathcal{H}}}
\renewcommand{\cL}{{\mathcal{L}}}
\newcommand{\cO}{{\mathcal{O}}}
\newcommand{\cP}{{\mathcal{P}}}

\newcommand{\cU}{\mathcal{U}}
\newcommand{\tU}{\tilde{\mathcal{U}}}
\newcommand{\tV}{\tilde{\mathcal{V}}}

\newcommand{\scp}[2]{\langle #1,#2\rangle}
\newcommand{\fl}[1]{\left\lfloor #1\right\rfloor}
\newcommand{\ce}[1]{\left\lceil #1\right\rceil}
\newcommand{\rcone}[1]{{{}_\Rgeo\!\left\langle #1\right\rangle}}
\newcommand{\zcone}[1]{{{}_\Zgeo\!\left\langle #1\right\rangle}}

\newcommand{\bn}{{\color{blue}\texttt{\bfseries n}}}
\newcommand{\bc}{{\color{blue}$\boldsymbol{\circ}$}}
\newcommand{\rn}{{\color{red}\texttt{\bfseries n}}}
\newcommand{\rs}{{\color{red}\texttt{\bfseries *}}}
\newcommand{\rt}{{\color{red}$\boldsymbol{\times}$}}
\newcommand{\bB}{{\color{blue}\texttt{\bfseries B}}}
\newcommand{\rR}{{\color{red}\texttt{\bfseries R}}}

\newcommand{\crb}[1]{{\color{red}\texttt{\bfseries{#1}}}}
\newcommand{\cbb}[1]{{\color{blue}\texttt{\bfseries{#1}}}}

\DeclareMathOperator{\interior}{int}
\DeclareMathOperator{\relint}{relint}
\DeclareMathOperator{\conv}{conv}
\DeclareMathOperator{\cone}{cone}
\DeclareMathOperator{\aff}{aff}
\DeclareMathOperator{\vol}{vol}
\DeclareMathOperator{\dist}{dist}
\DeclareMathOperator{\vertices}{vert}
\DeclareMathOperator{\rang}{rang}
\DeclareMathOperator{\sign}{sign}
\DeclareMathOperator{\pyr}{pyr}
\DeclareMathOperator{\bipyr}{bipyr}
\DeclareMathOperator{\Sl}{Sl}
\DeclareMathOperator{\im}{Im}
\DeclareMathOperator{\colspan}{colspan}
\DeclareMathOperator{\rank}{rank}

%% Albert
\DeclareMathOperator{\Ehr}{Ehr}

%%%%%%%%%%%%%%%%%%%%%%%%%%%%%%%%%%%%%%%%Ferran%%%%%%%%%%%%%%%%%%%%%%%%%%%%%%%%%%%%%%%%%%%%%%%%%%%%%%%%%%%%%%%%%%%%%%%%%%%%%%%
\newcommand{\Proj}{\mathbb{P}}
\DeclareMathOperator{\id}{id}
\DeclareMathOperator{\Vor}{Vor}
\newcommand{\HH}{\mathbb{H}}

\DeclareMathOperator{\sgn}{sgn} %signum
\DeclareMathOperator{\ggT}{ggT}

\newcommand{\ojo}[1]{\textsf{\bfseries\boldmath #1}}
\newcommand{\scribe}[1]{\begin{center}\emph{Scribe: #1}\end{center}\bigskip}

\graphicspath{{graphics/}}

\numberwithin{equation}{chapter}

\newtheorem{theorem}{\textbf{Theorem}}[chapter]
\newtheorem{lemma}[theorem]{Lemma}
\newtheorem{proposition}[theorem]{Proposition}
\newtheorem{corollary}[theorem]{Corollary}
\newtheorem{conj}[theorem]{Conjecture}
\newtheorem{exercise}[theorem]{Exercise}
\newtheorem{example}[theorem]{Example}
\newtheorem{claim}[theorem]{Claim}

\theoremstyle{definition}
\newtheorem{definition}[theorem]{Definition}
\newtheorem{defn}[theorem]{Definition}
\newtheorem{remark}[theorem]{Remark}
\newtheorem{observation}[theorem]{Observation}
\newtheorem{obs}[theorem]{Observation}

% \includeonly{lecture3}

\begin{document}

\thispagestyle{empty}

\newcommand{\thisyear}{2013}

\
\vfill
\begin{center}
        \Huge \sffamily\bfseries 
        Discrete and Algorithmic Geometry \thisyear
        \medskip
        (Part 2)

\vspace{2cm}
\LARGE
Julian Pfeifle

\vspace{3cm}

\normalfont\LARGE\sffamily
Version of \today

\vspace{5cm}\
\end{center}

This is the preliminary version of the lecture notes for the second
part of \emph{Discrete and Algorithmic Geometry} (Universitat
Polit�cnica de Catalunya), held in the fall semester of \thisyear\ by Ferran
Hurtado and Julian Pfeifle.

\medskip
These notes are fruit of the collaborative effort of all participating
students, who have taken turns in assembling this text. The name of
each scribe figures in each corresponding section.

\medskip
%The main literature for this course consists of
%\cite{Conway-Sloane-3rd}, \cite{Conway-Strauss08}
%and~\cite{Senechal95}. 

\medskip Suggestions for improvements will always be gladly received
by \texttt{julian.pfeifle@upc.edu}.

\vfill\


% Local Variables: 
% mode: latex
% TeX-master: "dag-upc"
% End: 

\tableofcontents

\chapter{Title of the lecture}

\scribe{Your name here}


% Local Variables: 
% mode: latex
% TeX-master: "dag-upc"
% End: 

\chapter{Asymptotic f-vectors of families of polytopes}

\scribe{Cecilia Gir\'on}

In this section we are going to study the \textit{unimodality conjecture} which says that there exists an $l = P(L)\in\mathbb{N}$ such that $f_0\leq f_1 \leq \cdots\leq f_l \leq f_{l+1}\leq \cdots \leq f_{d-1}\leq f_d$. We woould like to know if it is true. 

First, we define the \textbf{$f$-vector} as the vector of the form $(f_0,f_{1},\cdots,f_{d-1})$ where $f_i$ is as defined before in (\ref{eq1}). We will say that it is a \textbf{flag $f$-vector} $(f_s)_s = [d]$ such that $f_s$ count the number of flags $F_{i_1} \subset F_{i_2} \subset \cdots \subset F_{i_k}$ where $s = \{i_1,i_2,\cdots, i_k \}$ and $\dim F_{i_k} = i_k$ \footnote{You can also read about $cd$-index}.

\bigskip
The unimodal conjecture described before is known to be false for simplical polytopes of dimension $d\geq 19$ and for non-simplicial polytopes of dimension $d \geq 8$. The following conjecture is not known to be false. 

\textbf{Restricted unimodal conjecture (Anders Bjorner)}: 
\begin{eqnarray*}
 f_0\leq f_1\leq \cdots \leq f_{\lfloor \frac{d-1}{4}}\rfloor\\
 f_{\lfloor \frac{3(d-1)}{4}\rfloor}\geq \cdots \geq f_{d-1}
\end{eqnarray*} 

Intuitively we are sure that there is no way this conjecture could be false, but there is not proof of this. We don't even know if $f_k\geq \frac{1}{10000}\min\{f_0,f_{d-1}\}$ is true.

\bigskip

\textbf{Exercises done during the lecture 11/11/2013. Each one includes one}


\section{Operations on polytopes}
\begin{itemize}
\item \textbf{Cartesian (direct) product} $P\times Q$.
\item \textbf{Direct sum} $P^d \oplus q^e \subset \mathbb{R}^{d+e}$.
\item $P*Q\subset \mathbb{R}^{d+e+1}$. It is like $\oplus$ but the subspaces are skew (i.e. affine and they have no point on common). For example $\square^1 *\square^1 = Pyr(P)$.
\end{itemize}

\bigskip
\noindent\textsc{Example}: Given $f_k(P)$, calculate the $k$-th entry of $Pyr(P)$:
\begin{eqnarray*}
f(P)&=&(f_0,f_1,\cdots, f_{d-1}\\
f_k(Pyr(P)) &=& (f_0 +1, f_1 + f_0, f_2+f_1, \cdots, f_{d-2}+ f_{d-3}, f_{d-1}+ f_{d-2}, 1+ f_{d-1})
\end{eqnarray*}
\begin{flushright}
$\clubsuit$
\end{flushright}

\bigskip
\begin{itemize}
\item  \textbf{Connected sum} $P^d\#Q^d$ where $P$ has as simplicial face $f$ and $Q$ has a simplicial face $G$. 
\end{itemize}

This last operation is used to join the asymptotic function $f(\square^d)$ and its dual $f(\diamondsuit ^d)$. To make it work, since $\square^d$ has no triangulations in its faces, it is enough to cut away one vertex and, this way, get a simplex. Merging both functions using the connected sum gives place to a new function which is a non-unimodal function.


% Local Variables: 
% mode: latex
% TeX-master: "dag-upc"
% End: 

\chapter{Kalai's simple way of telling a simple polytope from its graph}

\scribe{Alex Alvarez}

\begin{conj}[Micha Perles, $\sim$1970]
Any $d$-polytope is determined by its graph, i.e., $sk^1(P)$ determines $\mathcal{F}(P)$.
\end{conj}

The conjecture is false, e.g. for simplicial polytopes. On the other hand, $K_n$ is dimensionally ambiguous, e.g. $sk^1(C_d(n)) = sk^1(C_e(n))$ for $d , e < n$. Besides there are neighborly graphs that are not combinatorially isomorphic to cyclic ones. However, the result is true for simple polytopes:

\begin{theorem}[Roswitha Blind, Peter Mani, 1987]
Any simple $d$-polytope is  determined by its graph.
\end{theorem}

Their proof is not constructive, but in 1988 Kalai found an algorithmic approach. Let us see the details of that approach.

\begin{definition}[Acyclic orientation]
An acyclic orientation of a graph $G$ is an orientation of $E(G)$ without directed cycles.
\end{definition}

\begin{definition}[Abstract Objective Function, AOF]
An AOF of $G = sk^1(P)$ is an acyclic orientation that has a unique sink on each face of $P$.
\end{definition}

\begin{definition}[Unique Sink Orientation]
A Unique Sink Orientation is an orientation such that every face has a unique sink. The orientation is not necessarily acyclic.
\end{definition}

Notice that AOFs exist because of the existence of generic linear objective functions. Consider the set of all AOFs and define the following function:

\begin{equation}
f(O) = h_0(O) + 2h_1(O) + \ldots + 2^kh_k(O) + \ldots + 2^dh_d(O)
\end{equation}
where $O$ is an AOF and $h_k(O)\in\mathbb{N}$ counts the vertices of in-degree $k$ in $O$.

\begin{figure}[!h]
\centering 
\centerline{%%Created by jPicEdt 1.4.1_03: mixed JPIC-XML/LaTeX format
%%Wed Dec 04 02:22:05 CET 2013
%%Begin JPIC-XML
%<?xml version="1.0" standalone="yes"?>
%<jpic x-min="0" x-max="40" y-min="0" y-max="40" auto-bounding="true">
%<parallelogram p3= "(40,0)"
%	 p2= "(40,40)"
%	 p1= "(0,40)"
%	 fill-style= "none"
%	 />
%<parallelogram p3= "(30,10)"
%	 p2= "(30,30)"
%	 p1= "(10,30)"
%	 fill-style= "none"
%	 />
%<multicurve arrow-bracket-length-scale= "0.25"
%	 arrow-global-scale-width= "1.5"
%	 fill-style= "none"
%	 arrow-rbracket-length-scale= "0.1"
%	 points= "(0,40);(0,40);(5,35);(5,35)"
%	 right-arrow= "head"
%	 arrow-head-width-minimum= "1.6"
%	 />
%<multicurve arrow-bracket-length-scale= "0.25"
%	 arrow-global-scale-width= "1.5"
%	 fill-style= "none"
%	 arrow-rbracket-length-scale= "0.1"
%	 points= "(0,0);(0,0);(0,20);(0,20)"
%	 right-arrow= "head"
%	 arrow-head-width-minimum= "1.6"
%	 />
%<multicurve arrow-bracket-length-scale= "0.25"
%	 arrow-global-scale-width= "1.5"
%	 fill-style= "none"
%	 arrow-rbracket-length-scale= "0.1"
%	 points= "(10,10);(10,10);(10,20);(10,20)"
%	 right-arrow= "head"
%	 arrow-head-width-minimum= "1.6"
%	 />
%<multicurve arrow-bracket-length-scale= "0.25"
%	 arrow-global-scale-width= "1.5"
%	 fill-style= "none"
%	 arrow-rbracket-length-scale= "0.1"
%	 points= "(0,0);(0,0);(5,5);(5,5)"
%	 right-arrow= "head"
%	 arrow-head-width-minimum= "1.6"
%	 />
%<multicurve arrow-bracket-length-scale= "0.25"
%	 arrow-global-scale-width= "1.5"
%	 fill-style= "none"
%	 arrow-rbracket-length-scale= "0.1"
%	 points= "(0,0);(0,0);(20,0);(20,0)"
%	 right-arrow= "head"
%	 arrow-head-width-minimum= "1.6"
%	 />
%<multicurve arrow-bracket-length-scale= "0.25"
%	 arrow-global-scale-width= "1.5"
%	 fill-style= "none"
%	 arrow-rbracket-length-scale= "0.1"
%	 points= "(40,0);(40,0);(35,5);(35,5)"
%	 right-arrow= "head"
%	 arrow-head-width-minimum= "1.6"
%	 />
%<multicurve arrow-bracket-length-scale= "0.25"
%	 arrow-global-scale-width= "1.5"
%	 fill-style= "none"
%	 arrow-rbracket-length-scale= "0.1"
%	 points= "(40,0);(40,0);(40,20);(40,20)"
%	 right-arrow= "head"
%	 arrow-head-width-minimum= "1.6"
%	 />
%<multicurve arrow-bracket-length-scale= "0.25"
%	 arrow-global-scale-width= "1.5"
%	 fill-style= "none"
%	 arrow-rbracket-length-scale= "0.1"
%	 points= "(30,10);(30,10);(30,20);(30,20)"
%	 right-arrow= "head"
%	 arrow-head-width-minimum= "1.6"
%	 />
%<multicurve arrow-bracket-length-scale= "0.25"
%	 arrow-global-scale-width= "1.5"
%	 fill-style= "none"
%	 arrow-rbracket-length-scale= "0.1"
%	 points= "(10,10);(10,10);(20,10);(20,10)"
%	 right-arrow= "head"
%	 arrow-head-width-minimum= "1.6"
%	 />
%<multicurve arrow-bracket-length-scale= "0.25"
%	 arrow-global-scale-width= "1.5"
%	 fill-style= "none"
%	 arrow-rbracket-length-scale= "0.1"
%	 points= "(10,30);(10,30);(20,30);(20,30)"
%	 right-arrow= "head"
%	 arrow-head-width-minimum= "1.6"
%	 />
%<multicurve arrow-bracket-length-scale= "0.25"
%	 arrow-global-scale-width= "1.5"
%	 fill-style= "none"
%	 arrow-rbracket-length-scale= "0.1"
%	 points= "(0,40);(0,40);(20,40);(20,40)"
%	 right-arrow= "head"
%	 arrow-head-width-minimum= "1.6"
%	 />
%<multicurve fill-style= "none"
%	 points= "(30,30);(30,30);(35,35);(35,35)"
%	 />
%<multicurve fill-style= "none"
%	 points= "(0,0);(0,0);(10,10);(10,10)"
%	 />
%<multicurve fill-style= "none"
%	 points= "(40,0);(40,0);(30,10);(30,10)"
%	 />
%<multicurve fill-style= "none"
%	 points= "(30,30);(30,30);(40,40);(40,40)"
%	 />
%<multicurve fill-style= "none"
%	 points= "(0,40);(0,40);(10,30);(10,30)"
%	 />
%<multicurve arrow-bracket-length-scale= "0.25"
%	 arrow-global-scale-width= "1.5"
%	 fill-style= "none"
%	 arrow-rbracket-length-scale= "0.1"
%	 points= "(30,30);(30,30);(35,35);(35,35)"
%	 right-arrow= "head"
%	 arrow-head-width-minimum= "1.6"
%	 />
%</jpic>
%%End JPIC-XML
%LaTeX-picture environment using emulated lines and arcs
%You can rescale the whole picture (to 80% for instance) by using the command \def\JPicScale{0.8}
\ifx\JPicScale\undefined\def\JPicScale{1}\fi
\unitlength \JPicScale mm
\begin{picture}(40,40)(0,0)
\linethickness{0.3mm}
\put(0,40){\line(1,0){40}}
\put(0,0){\line(0,1){40}}
\put(40,0){\line(0,1){40}}
\put(0,0){\line(1,0){40}}
\linethickness{0.3mm}
\put(10,30){\line(1,0){20}}
\put(10,10){\line(0,1){20}}
\put(30,10){\line(0,1){20}}
\put(10,10){\line(1,0){20}}
\linethickness{0.3mm}
\multiput(0,40)(0.12,-0.12){42}{\line(1,0){0.12}}
\put(5,35){\vector(1,-1){0.12}}
\linethickness{0.3mm}
\put(0,0){\line(0,1){20}}
\put(0,20){\vector(0,1){0.12}}
\linethickness{0.3mm}
\put(10,10){\line(0,1){10}}
\put(10,20){\vector(0,1){0.12}}
\linethickness{0.3mm}
\multiput(0,0)(0.12,0.12){42}{\line(1,0){0.12}}
\put(5,5){\vector(1,1){0.12}}
\linethickness{0.3mm}
\put(0,0){\line(1,0){20}}
\put(20,0){\vector(1,0){0.12}}
\linethickness{0.3mm}
\multiput(35,5)(0.12,-0.12){42}{\line(1,0){0.12}}
\put(35,5){\vector(-1,1){0.12}}
\linethickness{0.3mm}
\put(40,0){\line(0,1){20}}
\put(40,20){\vector(0,1){0.12}}
\linethickness{0.3mm}
\put(30,10){\line(0,1){10}}
\put(30,20){\vector(0,1){0.12}}
\linethickness{0.3mm}
\put(10,10){\line(1,0){10}}
\put(20,10){\vector(1,0){0.12}}
\linethickness{0.3mm}
\put(10,30){\line(1,0){10}}
\put(20,30){\vector(1,0){0.12}}
\linethickness{0.3mm}
\put(0,40){\line(1,0){20}}
\put(20,40){\vector(1,0){0.12}}
\linethickness{0.3mm}
\multiput(30,30)(0.12,0.12){42}{\line(1,0){0.12}}
\linethickness{0.3mm}
\multiput(0,0)(0.12,0.12){83}{\line(1,0){0.12}}
\linethickness{0.3mm}
\multiput(30,10)(0.12,-0.12){83}{\line(1,0){0.12}}
\linethickness{0.3mm}
\multiput(30,30)(0.12,0.12){83}{\line(1,0){0.12}}
\linethickness{0.3mm}
\multiput(0,40)(0.12,-0.12){83}{\line(1,0){0.12}}
\linethickness{0.3mm}
\multiput(30,30)(0.12,0.12){42}{\line(1,0){0.12}}
\put(35,35){\vector(1,1){0.12}}
\end{picture}
}

\caption{An example of AOF. For this orientation, $h_0 = 1$, $h_1 = 3$, $h_2 = 3$ and $h_3 = 1$. Therefore, $f(O) = 1 + 6 + 12 + 8 = 27 = 3^3$.}
\end{figure}

Let $f$ denote the number of non-empty faces of $P$. Given an acyclic orientation $O$, then

\begin{enumerate}
\item $f(O) \geq f$, because as $O$ is acyclic each face has at least one sink.
\item $f(O) = f$ if $O$ is an AOF.
\end{enumerate}

To determine all AOFs of $P$ just from its graph $G = sk^1(P)$ do the following steps:
\begin{enumerate}
\item Enumerate all acyclic orientations $O$ of $G$.
\item Calculate $f(O)$ for each such orientation $O$.
\item Keep the orientations $O$ with minimal $f(O)$.
\end{enumerate}

The only remaining step to complete Kalai's method is showing a way to identify the faces of $P$ from the knowledge of all the AOFs. Notice that the subgraphs of $G$ that are graphs of faces are connected, $k$-regular and induced. With that, the following proposition is enough to characterize the faces:

\begin{proposition}
An induced connected $k$-regular subgraph $H$ is the graph of a face of $P$ $\Leftrightarrow$ $H$ is an initial set for some AOF $O$.
\end{proposition}
\begin{proof}

$\Rightarrow$: Perturb a face-defining inequality to obtain a linear function with respect to which the vertices of $F$ lie below all other vertices.

$\Leftarrow$: Let $H$ be a connected $k$-regular subgraph of $sk^1(P)$ and let $O$ be an AOF such that $H$ is an initial set of $O$. As $H$ has a unique sink $v$ and is $k$-regular, the sink $v$ has $k$ incoming edges and as the polytope is simple, those $k$ edges determine a $k$-face $F$ and $v$ is a sink in that face.

As $v$ is the unique sink of $O$ in $F$, all vertices of $F$ are $\leq v$ with respect to $O$. But the initial set of $v$ in $G$ is $H$, so vert$(F) \subseteq $ vert$(H)$. As both $H$ and $F$ are connected and $k$-regular, we have that vert$(F) = $ vert$(H)$.
\end{proof}

Notice that this algorithm is exponential, but in 1995 Friedman gave a polynomial algorithm to reconstruct the 2-faces, which had been proven before to be enough to reconstruct the whole polytope.
 

% Local Variables: 
% mode: latex
% TeX-master: "dag-upc"
% End: 

\chapter{Ehrhart-Macdonald Reciprocity}

\scribe{Albert Santiago}

\section{Statement of the theorem}

Let $P$ be a lattice polytope. Let $L_P:\NN\longrightarrow\NN$ be the lattice point counting function,
\[\begin{array}{cccl}
  L_P: &\NN &\longrightarrow &\NN \\
  &t &\longmapsto &L_P(t)=\#\left\{tP\cap\ZZ^d\right\}
\end{array}\]

and let $L_{P^0}:\NN\longrightarrow\NN$ be the lattice point counting function in the interior of $P$.

By Ehrhart's Theorem seen in the previous lecture, it is known that $L_P, L_{P^0}\in\QQ[t]$.

\begin{theorem}[Ehrhart-Macdonald Reciprocity]
\[
  L_P(-t) = (-1)^d L_{P^0}(t)
\]
\end{theorem}

Note that in previous lecture we have already seen an example for this EM reciprocity for the cube. Indeed,
\begin{align*}
  L_{\Box^d}(t) &= (1+t)^d\\
  L_{(\Box^d)^0}(t) &= (-1)^d(1-t)^d
\end{align*}

\bigskip
\section{Auxiliary results}

\begin{proposition}
  Let $w_1,\ldots,w_d\in\ZZ^d$ be linearly independent. Let $C=\cone{w_1,\ldots,w_d}$ be cone over a simplex. Let $v\in\RR^d$ such that $\partial(v+C)\cap \ZZ^d=\varnothing$. Then,
  \[
    \sigma_{v+C}\left(\frac{1}{z_1},\cdots,\frac{1}{z_d}\right) = (-1)^d \sigma_{-v+C}(z_1,\ldots,z_d)
  \]
\end{proposition}

\emph{Notes}:
\begin{itemize}
 \item For the existence of such $v\in\RR^d$, see exercise sheet \#6.
 \item $\sigma_{v+C}\left(\frac{1}{z_1},\cdots,\frac{1}{z_d}\right)$ is not a power series, so the theorem is meaningless at this level. At most, we can say it is an element from $\QQ[z_1,\ldots,z_d]$, where the theorem holds.
 \item Try to see $\frac{1}{z_i}$ as the variables corresponding to $-P$.
\end{itemize}

\begin{proof}
  To be done. %TODO
\end{proof}

\begin{proposition}[Stanley Reciprocity]
  Let $C$ be a rational pointed cone (i. e. $C=\cone{w_1,\ldots,w_d}, w_i\in\QQ^d$) with apex at the origin. Then,
  \[
    \sigma_C\left(\frac{1}{z}\right)=(-1)^d\sigma_{C^0}(z)
  \]
\end{proposition}

\begin{proof}
  To be done. %TODO
\end{proof}

\begin{definition}
  We define the Ehrhart function for the interior lattice points of a polytope naturally as
  \[
    \Ehr_{P^0}(z):=\sum_{t\geq 1} L_{P^0}(t)z^t
  \]
\end{definition}

\emph{Note}: We can start the sum at $t=1$ rather than at $t=0$ because the origin is not an interior point of $P$.

\begin{observation}
  As it happened with the original Ehrhart function, the following equality holds:
  \[
    \Ehr_{P^0}(z)=\sigma_{\cone(P^0)}(1,\ldots,1,z)
  \]
\end{observation}

\begin{proposition}
  Let $P$ be a lattice polytope. Then,
  \[
    \Ehr_(P^0)(z)=(-1)^{\dim{P}+1} \Ehr_P\left(\frac{1}{z}\right)
  \]
\end{proposition}

\emph{Notes}:
\begin{itemize}
 \item The proposition is also valid for $P$ any \emph{rational} polytope, not only for lattice polytopes.
 \item Not every polytope can be perturbed into a rational one. There exist irrational polytopes.
\end{itemize}

\begin{proof}
  To be done. %TODO
\end{proof}

\bigskip
\section{Proof of Ehrhart-Macdonald Reciprocity}
To be done. %TODO

\bigskip
\section{Degree of a lattice polytope}

\begin{observation}
  Let $p\in\QQ[t]$ be a polynomial of degree $d$. If the following equality holds:
  \[
    \sum_{t\geq 0}p(t)z^t = \frac{h_dz^d+\cdots h_1z+h_0}{(1-z)^d}
  \]
  then
  \[
    \left.\begin{array}{r}
	     h_d=\cdots=h_{k+1}=0\\
	     h_k\neq0
          \end{array}\right\}
    \Longleftrightarrow
    \left\{\begin{array}{l}
             p(-1)=p(-2)=\cdots=p(-(d-k))=0\\
             p(-(d-k+1))\neq0
           \end{array}\right.
  \]
\end{observation}

\begin{proof}
  See Theorem 3.18 at Beck-Robins' \emph{Computing the continuous discretely}.
\end{proof}

\begin{proposition}
  Let $P\subset\RR^d$ be a lattice polytope with Ehrhart function:
  \[
    \Ehr_P(z) = \frac{h_dz^d+\cdots h_1z+h_0}{(1-z)^d}
  \]
  Then,
  \[
    \left.\begin{array}{r}
	     h_d=\cdots=h_{k+1}=0\\
	     h_k\neq0
          \end{array}\right\}
    \Longleftrightarrow
    \left\{\begin{array}{l}
             L_P(-1)=L_P(-2)=\cdots=L_P(-(d-k))=0\\
             L_P(-(d-k+1))\neq0
           \end{array}\right.
  \]
\end{proposition}

\begin{definition}
  We call $k$ the \emph{degree of the lattice polytope $P$}.
\end{definition}

\begin{observation}
  If $P$ is a $d$-polytope of degree $k$, then $L_P(-(d-k))=(-1)^d L_{P^0}(-(d-k))=0$, so the $(d-k)$-th dilate of $P$ does not contain any interior lattice point, and $(d-k+1)P$ is the smallest integer dilate that contains an interior lattice point.
\end{observation}

%TODO examples?


\bigskip 
\section{Reflexive polytopes}
\begin{definition}
  A lattice polytope $P$ with $0\in P$ is \emph{reflexive} if $P$ can be expressed as:
  \[
    P=\{x\in\RR^d:Ax\leq\mathbbm{1}\}
  \]
  where $A\in\ZZ^{n\times d}$ is an integer matrix.
\end{definition}

\begin{observation}
  To be done. %TODO
\end{observation}

To be ended. %TODO







\chapter*{Scribes 2013}

\section{Alex Alvarez}
I have done a double degree in Informatics Engineering and Mathematics at the UPC and during these years I have participated in programming contests both individually and representing the UPC. I am mainly interested in Algorithms and Data Structures and I would like to start a PhD in those topics next year, but I also enjoy learning about Discrete Mathematics, in particular Combinatorics and Graph Theory.


\section{Cecilia Gir\'on Albert}

I've got my Degree in Mathematics at the Universidad Aut\'onoma de Madrid, which I completed my fifth and sixth semesters 
at the University of Jyv\"askyla (Finland) as an Erasmus student. My degree has been mainly focused on subjects like 
analysis, statistics and numerical methods, although I am more interested in algebra and graph theory. 

I decided to study the Master in Advanced Mathematics and Mathematical Engineering to keep developing my mathematics skills 
in some theoretical subjects that can be applied in real life problems. Therefore, I believe that this course is a great 
opportunity to learn more about algorithmic and computing science and even more importantly, it may help me to find out 
what field I would like to focus on in the future.

\section{Anna Somoza}

I have recently finished a degree in Mathematics at the Universitat Polit\`ecnica de Catalunya. During this degree I developed a great interest in Algebra fields. In particular, I took the optional subjects \emph{Algebraic Geometry}, \emph{Algebraic Topology} and \emph{Galois Theory} and I wrote my Final Degree Thesis on a topic of Number Theory.

Now I'm taking the Master in Advanced Mathematics and Mathematical Engineering to develop my knowledge in these and other fields, and I my aim is to start a PhD in Number Theory next year. I enroled this subject because I have allways liked both computer science and geometry, and it seemed to be interesting. Therefore, I would be interested in the topic related to algebraic geometry.

\section{Daniel Torres}

My name is Daniel Torres and I am graduate in mathematics in UPC. Along the degree I have developed much interest in fields of topology, algebra and geometry, and some loathing to study (not to programming) numerical methods and modelling. I decided study this master, and particularly this subject, for expand my knowledge about my interests.

More concretely, I am doing this course with the hope it shows me about geometry.




% Local Variables: 
% mode: latex
% TeX-master: "dag-upc"
% End: 


\bibliographystyle{plainnat}
\bibliography{dag}

\renewcommand{\chaptername}{Chapter}
\chapter*{Papers to referee}


\bibliographystyle{amsplain}
\bibliography{papers-to-referee}

\nocite{donoho}
\nocite{eppstein-ziegler-fat-4-polytopes}
\nocite{friedman-graph-simple-polytope}
\nocite{gebert-sturmfels-theobald}
\nocite{haase-schicho}
\nocite{sanyal}
\nocite{schwartz-ziegler}
\nocite{sottile}
\nocite{szabo-welzl}
\nocite{welzl}

% Local Variables: 
% mode: latex
% TeX-master: "dag-upc"
% End: 


\end{document}

%%% Local Variables: 
%%% mode: latex
%%% TeX-master: t
%%% End: 
