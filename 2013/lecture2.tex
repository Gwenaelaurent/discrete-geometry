\chapter{Asymptotic f-vectors of families of polytopes}

\scribe{Cecilia Gir\'on}

\section{The Unimodality Conjecture}
In this section we are going to study the \textit{unimodality conjecture} which says that there exists an $l = P(L)\in\mathbb{N}$ such that $f_0\leq f_1 \leq \cdots\leq f_l \leq f_{l+1}\leq \cdots \leq f_{d-1}\leq f_d$. We woould like to know if it is true. 

First, we define the \textbf{$f$-vector} as the vector of the form $(f_0,f_{1},\cdots,f_{d-1})$ where $f_i$ is as defined before in (\ref{eq1}). We will say that it is a \textbf{flag $f$-vector} $(f_s)_s = [d]$ such that $f_s$ count the number of flags $F_{i_1} \subset F_{i_2} \subset \cdots \subset F_{i_k}$ where $s = \{i_1,i_2,\cdots, i_k \}$ and $\dim F_{i_k} = i_k$ \footnote{You can also read about $cd$-index}.

\bigskip
The unimodal conjecture described before is known to be false for simplical polytopes of dimension $d\geq 19$ and for non-simplicial polytopes of dimension $d \geq 8$. The following conjecture is not known to be false. 

\textbf{Restricted unimodal conjecture (Anders Bjorner)}: 
\begin{eqnarray*}
 f_0\leq f_1\leq \cdots \leq f_{\lfloor \frac{d-1}{4}}\rfloor\\
 f_{\lfloor \frac{3(d-1)}{4}\rfloor}\geq \cdots \geq f_{d-1}
\end{eqnarray*} 

Intuitively we are sure that there is no way this conjecture could be false, but there is not proof of this. We don't even know if $f_k\geq \frac{1}{10000}\min\{f_0,f_{d-1}\}$ is true.

\bigskip

\subsection{Exercises worked on during the lecture 11/11/2013. Each team includes one}

\begin{description}
\item[Exercise 1b (Alex Alvarez and Ivan Geffner)] \textit{Using the simple form $n! \approx \left(\frac{n}{e}\right)^n$ of Stirling's formula, show that $\phi_d(x) := \log\left(\frac{d}{xd}\right)$ is asymptotically proportional to $-x\log x - (1 - x)\log(1 - x)$, for $x\in (0,1)$ and $d \rightarrow \infty$. Discuss the real function $\phi_d$ on $[0,1]$.}

\bigskip
Using the simple form of Stirling's formula, we obtain the following form:

\begin{equation*}
{d \choose xd} \approx \frac{d^d}{xd^{xd}(d - xd)^{d - xd}}
\end{equation*}

So applying the logarithm, we get

\begin{eqnarray*}
\phi_d(x) &=& d\log d - xd\log xd - (d - xd)\log (d - xd) \\
&=& d\log d - xd\log x - xd\log d - (d - xd)(\log d + \log (1 - x)) \\
&=& d(-x\log x - (1 - x)\log (1 - x))
\end{eqnarray*}

Thus, the first part of the exercise is proven. Let us study now the shape of the function. 

The function clearly vanishes when $x$ tends to 0 and 1 and in this interval is non-negative. The first derivative of the function is

\begin{equation*}
\frac{d}{dx}(-(1-x) \log(1-x)-x \log(x)) = \log(1-x)-\log(x)
\end{equation*}

and therefore, there is only one point in which derivative vanishes, that is $x = \frac{1}{2}$. Now, the second derivative is 

\begin{equation*}
\frac{d^2}{dx^2}\log(1-x)-\log(x) = \frac{1}{x(x - 1)}
\end{equation*}

We can conclude then that the point is a maximum, so we have characterized the shape of the function.

\item[Exercise 2b (Borja and Cecilia)] \textit{Using the simple form $n!\approx \left(\frac{n}{e}\right)^n$ of Stirling's formula, show that $\psi_d(x) : = d(1-x)+ \log  \binom {d} {xd}$ is asymptotically proportional to $1-x-x\log x - (1-x)\log(1-x)$, where  $\log = \log_2$ denotes the binary logarithm. Find an approximation to the maximum of this function on $(0,1)$}.

\bigskip
For the first part of the exercise, by applying the Stirling's formula, in the binomial for of the given function:
\[\binom{d}{xd} = \left(\frac{d}{xd^x (d(1-x))^{1-x}}\right)^d\]

Then, substituting in $\psi_d(x)$:

\begin{eqnarray*}
\psi_d(x) &=& d(1-x) + \log \left(\frac{d}{xd^x (d(1-x))^{1-x}}\right)^d \\
&=& d(1-x) + d\left(\log{d} - x\log{x} - x\log{d} - (1-x)\log{d} - (1-x) \log{(1-x)}\right)\\
&=& d (1-x-x\log x - (1-x)\log(1-x))
\end{eqnarray*}
Hence, $\psi_d(x)$ is asymptotically proportional to $1-x-x\log x - (1-x)\log(1-x)$
\bigskip

For the second part of the exercise, in order to find the maximum of the function
\begin{equation*}
f(x)=1-x-x\log(x)-(1-x)\log(1-x)
\end{equation*}
in (0,1) we will the points that have first derivative equal to 0, that is $x$ such that $f'(x)=0$, and computing $f'(x)$ we get:
\begin{equation*}
f'(x)=-1-(\log x+1)-(-\log(1-x)-1)=\log\left(\frac{1-x}{x}\right)-1
\end{equation*}
Now the points that make $f'(x)=0$ are the ones that make $\log(\frac{1-x}{x})=1$, which is the same as $x$ such that $\frac{1-x}{x}=e$, which translates into:
\begin{equation*}
x_{\max}=\frac{1}{e+1}
\end{equation*}
The shape of this function is a growing function from $x=0$ starting at $f(0)=1$ to $x=x_{\max}$, where it gets it's maximum, that is approximated by $f(x_max)\approx1.0414$  and then decreases to $0$ at $x=1$.
\end{description}

\section{Operations on polytopes}
\begin{itemize}
\item \textbf{Cartesian (direct) product} $P\times Q$.
\item \textbf{Direct sum} $P^d \oplus q^e \subset \mathbb{R}^{d+e}$.
\item $P*Q\subset \mathbb{R}^{d+e+1}$. It is like $\oplus$ but the subspaces are skew (i.e. affine and they have no point on common). For example $\square^1 *\square^1 = Pyr(P)$.
\end{itemize}

\bigskip
\noindent\textsc{Example}: Given $f_k(P)$, calculate the $k$-th entry of $Pyr(P)$:
\begin{eqnarray*}
f(P)&=&(f_0,f_1,\cdots, f_{d-1}\\
f_k(Pyr(P)) &=& (f_0 +1, f_1 + f_0, f_2+f_1, \cdots, f_{d-2}+ f_{d-3}, f_{d-1}+ f_{d-2}, 1+ f_{d-1})
\end{eqnarray*}
\begin{flushright}
$\clubsuit$
\end{flushright}

\bigskip
\begin{itemize}
\item  \textbf{Connected sum} $P^d\#Q^d$ where $P$ has as simplicial face $f$ and $Q$ has a simplicial face $G$. 
\end{itemize}

This last operation is used to join the asymptotic function $f(\square^d)$ and its dual $f(\diamondsuit ^d)$. To make it work, since $\square^d$ has no triangulations in its faces, it is enough to cut away one vertex and, this way, get a simplex. Merging both functions using the connected sum gives place to a new function which is a non-unimodal function.


% Local Variables: 
% mode: latex
% TeX-master: "dag-upc"
% End: 
