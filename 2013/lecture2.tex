\chapter{Asymptotic f-vectors of families of polytopes}

\scribe{Cecilia Gir\'on}

In this section we are going to study the \textit{unimodality conjecture} which says that there exists an $l = P(L)\in\mathbb{N}$ such that $f_0\leq f_1 \leq \cdots\leq f_l \leq f_{l+1}\leq \cdots \leq f_{d-1}\leq f_d$. We woould like to know if it is true. 

First, we define the \textbf{$f$-vector} as the vector of the form $(f_0,f_{1},\cdots,f_{d-1})$ where $f_i$ is as defined before in (\ref{eq1}). We will say that it is a \textbf{flag $f$-vector} $(f_s)_s = [d]$ such that $f_s$ count the number of flags $F_{i_1} \subset F_{i_2} \subset \cdots \subset F_{i_k}$ where $s = \{i_1,i_2,\cdots, i_k \}$ and $\dim F_{i_k} = i_k$ \footnote{You can also read about $cd$-index}.

\bigskip
The unimodal conjecture described before is known to be false for simplical polytopes of dimension $d\geq 19$ and for non-simplicial polytopes of dimension $d \geq 8$. The following conjecture is not known to be false. 

\textbf{Restricted unimodal conjecture (Anders Bjorner)}: 
\begin{eqnarray*}
 f_0\leq f_1\leq \cdots \leq f_{\lfloor \frac{d-1}{4}}\rfloor\\
 f_{\lfloor \frac{3(d-1)}{4}\rfloor}\geq \cdots \geq f_{d-1}
\end{eqnarray*} 

Intuitively we are sure that there is no way this conjecture could be false, but there is not proof of this. We don't even know if $f_k\geq \frac{1}{10000}\min\{f_0,f_{d-1}\}$ is true.

\bigskip

\textbf{Exercises done during the lecture 11/11/2013. Each one includes one}


\section{Operations on polytopes}
\begin{itemize}
\item \textbf{Cartesian (direct) product} $P\times Q$.
\item \textbf{Direct sum} $P^d \oplus q^e \subset \mathbb{R}^{d+e}$.
\item $P*Q\subset \mathbb{R}^{d+e+1}$. It is like $\oplus$ but the subspaces are skew (i.e. affine and they have no point on common). For example $\square^1 *\square^1 = Pyr(P)$.
\end{itemize}

\bigskip
\noindent\textsc{Example}: Given $f_k(P)$, calculate the $k$-th entry of $Pyr(P)$:
\begin{eqnarray*}
f(P)&=&(f_0,f_1,\cdots, f_{d-1}\\
f_k(Pyr(P)) &=& (f_0 +1, f_1 + f_0, f_2+f_1, \cdots, f_{d-2}+ f_{d-3}, f_{d-1}+ f_{d-2}, 1+ f_{d-1})
\end{eqnarray*}
\begin{flushright}
$\clubsuit$
\end{flushright}

\bigskip
\begin{itemize}
\item  \textbf{Connected sum} $P^d\#Q^d$ where $P$ has as simplicial face $f$ and $Q$ has a simplicial face $G$. 
\end{itemize}

This last operation is used to join the asymptotic function $f(\square^d)$ and its dual $f(\diamondsuit ^d)$. To make it work, since $\square^d$ has no triangulations in its faces, it is enough to cut away one vertex and, this way, get a simplex. Merging both functions using the connected sum gives place to a new function which is a non-unimodal function.


% Local Variables: 
% mode: latex
% TeX-master: "dag-upc"
% End: 
