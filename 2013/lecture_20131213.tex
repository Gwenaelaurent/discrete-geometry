\chapter{Ehrhart-Macdonald Reciprocity}

\scribe{Albert Santiago}

\section{Statement of the theorem}

Let $P$ be a lattice polytope. Let $L_P:\NN\longrightarrow\NN$ be the lattice point counting function,
\[\begin{array}{cccl}
  L_P: &\NN &\longrightarrow &\NN \\
  &t &\longmapsto &L_P(t)=\#\left\{tP\cap\ZZ^d\right\}
\end{array}\]

and let $L_{P^0}:\NN\longrightarrow\NN$ be the lattice point counting function in the interior of $P$.

By Ehrhart's Theorem seen in the previous lecture, it is known that $L_P, L_{P^0}\in\QQ[t]$.

\begin{theorem}[Ehrhart-Macdonald Reciprocity]
\[
  L_P(-t) = (-1)^d L_{P^0}(t)
\]
\end{theorem}

Note that in previous lecture we have already seen an example for this EM reciprocity for the cube. Indeed,
\begin{align*}
  L_{\Box^d}(t) &= (1+t)^d\\
  L_{(\Box^d)^0}(t) &= (-1)^d(1-t)^d
\end{align*}

\bigskip
\section{Auxiliary results}

\begin{proposition}
  Let $w_1,\ldots,w_d\in\ZZ^d$ be linearly independent. Let $C=\cone{w_1,\ldots,w_d}$ be cone over a simplex. Let $v\in\RR^d$ such that $\partial(v+C)\cap \ZZ^d=\varnothing$. Then,
  \[
    \sigma_{v+C}\left(\frac{1}{z_1},\cdots,\frac{1}{z_d}\right) = (-1)^d \sigma_{-v+C}(z_1,\ldots,z_d)
  \]
\end{proposition}

\emph{Notes}:
\begin{itemize}
 \item For the existence of such $v\in\RR^d$, see exercise sheet \#6.
 \item $\sigma_{v+C}\left(\frac{1}{z_1},\cdots,\frac{1}{z_d}\right)$ is not a power series, so the theorem is meaningless at this level. At most, we can say it is an element from $\QQ[z_1,\ldots,z_d]$, where the theorem holds.
 \item Try to see $\frac{1}{z_i}$ as the variables corresponding to $-P$.
\end{itemize}

\begin{proof}
  To be done. %TODO
\end{proof}

\begin{proposition}[Stanley Reciprocity]
  Let $C$ be a rational pointed cone (i. e. $C=\cone{w_1,\ldots,w_d}, w_i\in\QQ^d$) with apex at the origin. Then,
  \[
    \sigma_C\left(\frac{1}{z}\right)=(-1)^d\sigma_{C^0}(z)
  \]
\end{proposition}

\begin{proof}
  To be done. %TODO
\end{proof}

\begin{definition}
  We define the Ehrhart function for the interior lattice points of a polytope naturally as
  \[
    \Ehr_{P^0}(z):=\sum_{t\geq 1} L_{P^0}(t)z^t
  \]
\end{definition}

\emph{Note}: We can start the sum at $t=1$ rather than at $t=0$ because the origin is not an interior point of $P$.

\begin{observation}
  As it happened with the original Ehrhart function, the following equality holds:
  \[
    \Ehr_{P^0}(z)=\sigma_{\cone(P^0)}(1,\ldots,1,z)
  \]
\end{observation}

\begin{proposition}
  Let $P$ be a lattice polytope. Then,
  \[
    \Ehr_(P^0)(z)=(-1)^{\dim{P}+1} \Ehr_P\left(\frac{1}{z}\right)
  \]
\end{proposition}

\emph{Notes}:
\begin{itemize}
 \item The proposition is also valid for $P$ any \emph{rational} polytope, not only for lattice polytopes.
 \item Not every polytope can be perturbed into a rational one. There exist irrational polytopes.
\end{itemize}

\begin{proof}
  To be done. %TODO
\end{proof}

\bigskip
\section{Proof of Ehrhart-Macdonald Reciprocity}
To be done. %TODO

\bigskip
\section{Degree of a lattice polytope}

\begin{observation}
  Let $p\in\QQ[t]$ be a polynomial of degree $d$. If the following equality holds:
  \[
    \sum_{t\geq 0}p(t)z^t = \frac{h_dz^d+\cdots h_1z+h_0}{(1-z)^d}
  \]
  then
  \[
    \left.\begin{array}{r}
	     h_d=\cdots=h_{k+1}=0\\
	     h_k\neq0
          \end{array}\right\}
    \Longleftrightarrow
    \left\{\begin{array}{l}
             p(-1)=p(-2)=\cdots=p(-(d-k))=0\\
             p(-(d-k+1))\neq0
           \end{array}\right.
  \]
\end{observation}

\begin{proof}
  See Theorem 3.18 at Beck-Robins' \emph{Computing the continuous discretely}.
\end{proof}

\begin{proposition}
  Let $P\subset\RR^d$ be a lattice polytope with Ehrhart function:
  \[
    \Ehr_P(z) = \frac{h_dz^d+\cdots h_1z+h_0}{(1-z)^d}
  \]
  Then,
  \[
    \left.\begin{array}{r}
	     h_d=\cdots=h_{k+1}=0\\
	     h_k\neq0
          \end{array}\right\}
    \Longleftrightarrow
    \left\{\begin{array}{l}
             L_P(-1)=L_P(-2)=\cdots=L_P(-(d-k))=0\\
             L_P(-(d-k+1))\neq0
           \end{array}\right.
  \]
\end{proposition}

\begin{definition}
  We call $k$ the \emph{degree of the lattice polytope $P$}.
\end{definition}

\begin{observation}
  If $P$ is a $d$-polytope of degree $k$, then $L_P(-(d-k))=(-1)^d L_{P^0}(-(d-k))=0$, so the $(d-k)$-th dilate of $P$ does not contain any interior lattice point, and $(d-k+1)P$ is the smallest integer dilate that contains an interior lattice point.
\end{observation}

%TODO examples?


\bigskip 
\section{Reflexive polytopes}
\begin{definition}
  A lattice polytope $P$ with $0\in P$ is \emph{reflexive} if $P$ can be expressed as:
  \[
    P=\{x\in\RR^d:Ax\leq\mathbbm{1}\}
  \]
  where $A\in\ZZ^{n\times d}$ is an integer matrix.
\end{definition}

\begin{observation}
  To be done. %TODO
\end{observation}

To be ended. %TODO






