\documentclass[11pt]{amsart}

\usepackage{a4wide}
\usepackage{paralist}
\usepackage{url}
\usepackage{nopageno}

\newcommand{\cA}{\mathcal{A}}
\newcommand{\cS}{\mathcal{S}}

\begin{document}
\begin{center}
\textbf{\sffamily
   Discrete and Algorithmic Geometry }

\medskip
   Julian Pfeifle,
   UPC, 2014 \mbox{}
\end{center}

\bigskip

\begin{center}
  \textbf{\sffamily Participants}
\end{center}

\medskip

\section*{Quentin Pergeline}
(github login lafaillette) I am a french engineer with some knowledge in computer science and applied mathematics.
I am currently doing an Erasmus semester at the UPC FME, before trying to find an internship to get my diploma.
\medskip

\section*{Clara Mateo}
I am from Santa Perp\`etua de la Mogoda but I live in Barcelona since 2008. I studied the double degree in Mathematics and Physics in UAB. I started the master last year and I plan to finish it in June. About my professional experience, I did a Summer Scholarship in DESY, Hamburg, in 2012 and I have also worked in marketing online, creating and managing SEM in Google Adwords. Nowadays I am trying to figure out what I would like to do in the future :)
\medskip


\section*{Jan Fiser}
I come from the Czech Republic. Last year I finished bachelor's degree in "Mathematical Methods of Information Security" at the Charles University in Prague and this academic year I am studying at UPC with the Erasmus programme.
Although I am interested more in cryptography and abstract algebra, I enrolled here, among others, for courses of discrete algebra or graph theory. I am discovering new areas of mathematics and I enjoy it.
\medskip

\section*{Arnau Planas}
(github login arnauplanas) I studied a degree in Mathematics and another one in Physical Engineering at UPC (CFIS), 
and I'm coursing this master with the aim to do a PhD. 
I'm specially interested in the area of Differential Geometry and Topology.
\medskip

\section*{Irene de Parada}

I am from Madrid, where I studied Mathematical Engineering at the Complutense University. While doing this I started my bachelor in Physics  at the National University of Distance Education (UNED) and I am currently finishing it. Last year I took some courses at KTH (Stockholm) on statistical methods in Computer Science and Economy. My  bachelor's thesis was about UAV route optimization problem  among obstacles in 2D and 3D. My main areas of interest are Discrete Geometry and Optimization and I would like my future to be related with them.

\section*{Aitor Perez}
7 years ago I started the double bachelor's degree in Mathematics and Computer Science in UPC (CFIS). Two years ago, I finished the degree in Mathematics, and last year the degree in Computer Science. I've been working for the last two years, in a big advisory company and now in a very small start-up, as a web developer. In both of them I missed Maths and so I decided to register for this master and maybe continue with a PhD.

\section*{Ainize Cidoncha}
I am from the Basque Country where I studied the first three years of my degree in Mathematics. Last year I spent a year as an Erasmus student in The Netherlands at Utrecht University where I finished my degree. 
Currrently I am working part time at Bayer Hipania SL as a trainee in an efficiency project.

\section*{Elisabet Burjons Pujol}

I am from Pineda de Mar, a town near Barcelona. I have studied Mathematics at UPC and Physics at UB. 
I am taking this master in order to learn more about algorithms and discrete mathematics.
My goal is to PhD in applied mathematics or theoretical computer science if possible.

My topics of interest are online algorithms, randomized algorithms and graphs,
but I like in general all kinds of discrete mathematics and combinatorics related topics.

\section*{Cassandra de Cock de Rameyen}
I am from Waterloo in Belgium, I am currently in my second year of master in Engineering in Applied Mathematics at the Université Catholique de Louvain. I also did my bachelor in this university. I am interested in every thing that is related to the mathematics, I have already done some optimisation, operation research, graph theory, algorithmic and a little bit of modelisation.

\end{document}
