\documentclass[11pt]{amsart}

\usepackage{a4wide}
\usepackage{paralist}
\usepackage{url}
\usepackage{nopageno}

\newcommand{\cA}{\mathcal{A}}
\newcommand{\cS}{\mathcal{S}}

\begin{document}
\begin{center}
\textbf{\sffamily
   Discrete and Algorithmic Geometry }

\medskip
   Julian Pfeifle,
   UPC, 2014 \mbox{}
\end{center}

\bigskip

\begin{center}
  \textbf{\sffamily Sheet 1}

\bigskip
 due on Monday, November 10, 2014

\end{center}

\bigskip
\bigskip
\bigskip

\section*{Reading}

\begin{enumerate}
\setlength{\itemsep}{2ex}
\item Read Lectures 0,1,2 from Ziegler's \emph{Lectures on Polytopes}.

\item Read Sections 5.1, 5.2, 5.3 from Matou\v sek's \emph{Lectures on
    Discrete Geometry}.

\end{enumerate}

\bigskip
\bigskip
\section*{Writing}

\begin{enumerate}
\item Let $P$ be a $3$-dimensional polytope with $n$
  vertices. 
  Prove or disprove: 
  If the graph of~$P$ is complete (i.e., between every pair of points
  there is an edge on the convex hull), then $n=4$, so that $P$~is a tetrahedron.
\end{enumerate}

\bigskip
\bigskip
\section*{Software}

\begin{enumerate}
\setlength{\itemsep}{2ex}
\item Create a \texttt{github} account, clone the repository
  \begin{center}
    \url{https://github.com:julian-upc/discrete-geometry.git} ,
  \end{center}
  and write around two paragraphs presenting yourself into the file
  \texttt{2014/participants.tex}~. Commit your edits, push your
  commits to your clone, and issue a pull request so that your edits
  can be merged into the central repository.

\item Create a public/private key pair for \texttt{gpg} and place the
  public key into \texttt{2014/public\_keys}~. Again, don't forget to
  commit and push your changes, and issue a pull request.

\item Install \texttt{polymake 2.13} from \url{polymake.org} on your
  computer. and play around with it. For example, try to understand
  the (interactive) drawings that result from saying
  \texttt{cube(4)->VISUAL;} and
  \texttt{regular\_120\_cell()->VISUAL;}~. Try to make sense of the
  output of \texttt{cube(3)->properties();}~. The command
  \texttt{help("YourTopicHere")} is your friend, as is the
  documentation on \texttt{polymake.org}.
\end{enumerate}

\end{document}
