\setcounter{chapter}{7}
\chapter{Introduction to orbifolds}
\scribe{Ane Santos}

\begin{definition}
Informally, an \emph{orbifold} is the quotient of a manifold (here, the Euclidean plane) by the action of a group.
\end{definition}

\begin{center}
	\begin{tabular}{lcc}
		torus & $\longleftrightarrow$ & \bc \\
		holes & $\longleftrightarrow$ & \rs \\
		non-orientability & $\longleftrightarrow$ & \rt \\
		boundary singularity & $\longleftrightarrow$ & \rs\rn \\
		cone point of order $n$ & $\longleftrightarrow$ & \bn \
	\end{tabular}
\end{center}

\begin{theorem}[Magic theorem for the sphere]
The total cost of the signature of any spherical group is $ \$ 2-\frac{2}{g}$, where
$g$~denotes the total number of symmetries.
\end{theorem}

The Magic theorem in the plane is a special case because the number of symmetries in a plane is infinite, so the cost is always 2.

There are 14 spherical symmetry groups: $m,n\geq1$

\begin{center}
	\begin{tabular}{lcccc}
		\crb{*532} & \crb{*432} & \crb{*332} & \crb{*22n} & \crb{*mn} \\
		 & & \cbb{3}\crb{*2} & \cbb{2}\crb{*n} & \cbb{n}\crb{*} \\
		 & & & & \cbb{n}\rt \\
		 \cbb{532} & \cbb{432} & \cbb{332} & \cbb{22n} & \cbb{mn} 
	\end{tabular}
\end{center}

If $n\rightarrow\infty$ and $m\rightarrow\infty$ in \crb{*22n}, \crb{*mn}, \cbb{2}\crb{*n}, \cbb{n}\rs, \bn\rt, \cbb{22n} and \cbb{mn} we get the 7 possible groups of friezes (cenefas).

\bigskip
We spent almost the entire lecture with scissors and tape, cutting out the orbifolds corresponding to the
tesselations in~\cite{Conway-Strauss08}.


% Local Variables: 
% mode: latex
% TeX-master: "dag-upc"
% End: 
